\documentclass[11pt]{article} % use larger type; default would be 10pt

\usepackage{graphicx} % support the \includegraphics command and options
\usepackage[margin=1.0in]{geometry}
\usepackage{amssymb}
\usepackage{amsmath}

\begin{document}

\begin{center}
\textbf{Quadratically Complete PEM Elements:}
\end{center}

In the following summary, a ``new'' method for constructing quadratically complete PEM shape functions is proposed. In some ways, this approach can be viewed as a simplification of the ``old'' (more expensive) method which was outlined during my qualifying examination.

To provide some context, recall that the ``old'' method allowed for the shape functions to vary quadratically within each quadrature cell, i.e. within cell $\omega_p$ the shape functions would vary according to:
\begin{equation}
	\varphi_p (\mathbf{x}) = c_p + (\mathbf{x} - \mathbf{x}_p) \cdot \mathbf{g}_p + \left[ (\mathbf{x} - \mathbf{x}_p) \otimes (\mathbf{x} - \mathbf{x}_p) \right] \colon \mathbf{H}_p,
\end{equation}
where $\mathbf{H}_p$ is a symmetric rank-2 tensor of ``curvature coefficients,'' supplementing the cell's original constant and gradient coefficients ($c_p$ and $\mathbf{g}_p$, respectively).

The key benefit of this approach is that the resulting shape functions are truly quadratically complete, owing to their piece-wise quadratic representation. However, one of the major drawbacks of this approach is that it requires the inclusion of an excessive number of additional unknown curvature coefficients (in three dimensions, $\mathbf{H}_p$ introduces 6 additional unknown terms \textit{for each cell} $\omega_p$.) Consequently, the linear system which arises from the minimum problem becomes enormous in size. Moreover, it was discovered through experimentation that an additional ``jump in curvature'' term had to be added to the PEM functional in order to control apparent instabilities/oscillations in the resulting shape functions.

In spite of these difficulties, the method at least works. What's more, it may be amenable to a very simple adjustment which could obviate some of the aforementioned issues, leading to the ``new'' method: rather than allowing for each cell to possess its own set of independent curvature coefficients, what if we were to impose an additional set of constraints on the curvature terms such that
\begin{equation}
	\mathbf{H}_p - \mathbf{H}_q = \mathbf{0} \, \, \, \forall p, \, q.
\end{equation}
In other words: make the curvature coefficients the \textit{same} across all cells ($\mathbf{H}_p = \bar{\mathbf{H}} \, \, \forall p$), thus necessitating only a flat 6 additional unknowns for each minimum problem. As an added benefit, we would no longer need to include the auxiliary ``jump in curvature'' term in the functional (being that all jumps in curvature are \textit{enforced} to be zero).

Still, a number of concerns remain. Since the resulting shape functions of this ``new'' method are piece-wise quadratic, it is not clear how the numerical integration of such functions should be handled. The interpretation of quadrature consistency in this context would need to be made clear. Additionally, we would still need to carry out the integration of cell monomial moments up to 4\textsuperscript{th} order, as these would appear in the ``jump in value'' terms of the PEM functional.

Nonetheless, the method's reduced computational expense may justify further exploration. Perhaps the approach could even be extended to account for a variable number of constraints amongst the cell curvature terms, yielding a family of such methods which exhibit uniform curvature over sub-groupings of cells.

\end{document}