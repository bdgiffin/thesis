%\documentclass[12pt,draftcls]{ucdavisthesis}
\documentclass[12pt]{ucdavisthesis}

% PLEASE READ THE MANUAL - ucdavisthesis.pdf (in the package installation directory)

%%%%%%%%%%%%%%%%%%%%%%%%%%%%%%%%%%%%%%%%%%%%%%%%%%%%%%%%%%%%%%%%%%%%%%%%
%                                                                      %
%               LATEX COMMANDS FOR DOCUMENT SETUP                      %
%                                                                      %
%%%%%%%%%%%%%%%%%%%%%%%%%%%%%%%%%%%%%%%%%%%%%%%%%%%%%%%%%%%%%%%%%%%%%%%%

%\usepackage{bookmark}
\usepackage[us,nodayofweek,12hr]{datetime}
\usepackage{graphicx}
%\usepackage[square,comma,numbers,sort&compress]{natbib}
%\usepackage{hypernat}
% Other useful packages to try
%\usepackage{amsmath}
%\usepackage{amssymb}
%
% Different fonts to try (uncomment only fontenc and one font at a time)
% (you may need to install these first)
%\usepackage[T1]{fontenc} %enable fontenc package if using one of the fonts below
%\usepackage[adobe-utopia]{mathdesign}
%\usepackage{tgschola}
%\usepackage{tgbonum}
%\usepackage{tgpagella}
%\usepackage{tgtermes}
%\usepackage{fourier}
%\usepackage{fouriernc}
%\usepackage{kmath,kerkis}
%\usepackage{kpfonts}
%\usepackage[urw-garamond]{mathdesign}
%\usepackage[bitstream-charter]{mathdesign}
%\usepackage[sc]{mathpazo}
%\usepackage{mathptmx}
%\usepackage[varg]{txfonts}

\hyphenation{dis-ser-ta-tion blue-print man-u-script pre-par-ing} %add hyphenation rules for words TeX doesn't know


%\renewcommand{\rightmark}{\scriptsize A University of California Davis\ldots \hfill Rev.~\#1.0 \quad Compiled: \currenttime, \today}
% a fancier running header that can be used with draftcls options

%%%%%%%%%%%%%%%%%%%%%%%%%%%%%%%%%%%%%%%%%%%%%%%%%%%%%%%%%%%%%%%%%%%%%%%%
%                                                                      %
%        DOCUMENT SETUP AND INFORMATION FOR PRELIMINARY PAGES          %
%                                                                      %
%%%%%%%%%%%%%%%%%%%%%%%%%%%%%%%%%%%%%%%%%%%%%%%%%%%%%%%%%%%%%%%%%%%%%%%%

\title          {Thesis\\
                 Title}
%Exact title of your thesis. Indicate italics where necessary by underlining or using italics. Please capitalize the first letter of each word that would normally be capitalized in a title.

\author         {Brian Doran Giffin}
%Your full name as it appears on University records. Do not use initials.

\authordegrees  {B.S. (University of California, Davis) 2013 \\
                 M.S. (University of California, Davis) 2014}
%Indicate your previous degrees conferred.

\officialmajor  {\textsc{Civil and Environmental Engineering}}
%This is your official major as it appears on your University records.

\graduateprogram{Civil and Environmental Engineering}
%This is your official graduate program name. Used for UMI abstract.

\degreeyear     {2017}
% Indicate the year in which your degree will be officially conferred.

\degreemonth    {December}
% Indicate the month in which your degree will be officially conferred. Used for UMI abstract.

\committee{\textsc{Mark M. Rashid, Chair}}{\textsc{N. Sukumar}}{\textsc{Boris Jeremi\'{c}}}{}{}
% These are your committee members. The command accepts up to five committee members so be sure to have five sets of braces, even if there are empties.

%%%%%%%%%%%%%%%%%%%%%%%%%%%%%%%%%%%%%%%%%%%%%%%%%%%%%%%%%%%%%%%%%%%%%%%%

%\copyrightyear{2020}
%\nocopyright

%%%%%%%%%%%%%%%%%%%%%%%%%%%%%%%%%%%%%%%%%%%%%%%%%%%%%%%%%%%%%%%%%%%%%%%%

\dedication{\textsl{To Lyle \ldots \\
            for teaching me how think deeply, and to appreciate the nuance in all things.} \\
            \textsl{And to Mark \ldots \\
            for teaching me not to think for too long -- to take action, instead.}}

%%%%%%%%%%%%%%%%%%%%%%%%%%%%%%%%%%%%%%%%%%%%%%%%%%%%%%%%%%%%%%%%%%%%%%%%

\abstract{A high-level discussion of the fundamental problems accompanying finite element discretizations in the form of locking, addressing the scope of the work under consideration.}

%%%%%%%%%%%%%%%%%%%%%%%%%%%%%%%%%%%%%%%%%%%%%%%%%%%%%%%%%%%%%%%%%%%%%%%%

\acknowledgments{Acknowledgements to those who helped you get to this point. They should be listed by chapter when appropriate.}

%%%%%%%%%%%%%%%%%%%%%%%%%%%%%%%%%%%%%%%%%%%%%%%%%%%%%%%%%%%%%%%%%%%%%%%%

% Each chapter can be in its own file for easier editing and brought in with the \include command.
% Then use the \includeonly command to speed compilation when working on a particular chapter.
%%% \includeonly{ucdavisthesis_example_Chap1}

\begin{document}

\newcommand{\bibfont}{\singlespacing}
% need this command to keep single spacing in the bibliography when using natbib

\bibliographystyle{plain}
%many other bibliography styles are available (IEEEtran, mla, etc.). Use one appropriate for your field.

\makeintropages %Processes/produces the preliminary pages

\chapter{Historical Perspectives}
%
A survey of various approaches and methods to combat the inherrent issue of locking, told chronologically.

\section{Discretization Methods}
%
An overview of various approaches to the problem of discretizing a solid domain for the purposes of obtaining a computational mesh amenable to a particular approximation scheme.
\subsection{Discretizations with Regular Shapes}

\subsection{Discretizations with Arbitrary Polytopes}
\subsubsection{Voronoi Diagrams}
\subsubsection{Cut Cell Methods}

\subsection{Mesh Quality Metrics}

\section{Approximation Schemes}
%
A detailed discussion of various approximation schemes in common usage, and an evaluation of their performance in the context of non-linear solid mechanics.
\subsection{Continuous Galerkin Methods}
\subsubsection{Structural Finite Elements}

\subsection{Discontinuous Galerkin Methods}

\subsection{Mesh Free Methods}

\subsection{Weak Galerkin Method}

\subsection{Virtual Element Method}

\chapter{Mathematical Preliminaries}
%
Discuss the mathematical problems being solved for solids mechanics, etc., and cover some fundamental notational aspects. Also, talk about characterization of locking phenomena from a rigorous mathematical perspective.

\section{Notation}

\section{}

\chapter{Considerations for Computational Solid Mechanics}
%
Justify a methodology that suitably addresses all of the issues discussed herein. A method which could solve all of these problems effectively likely does not exist, but may be ideal for a particular application.

\section{Enforcement of Constraints}
\subsection{Essential Boundary Conditions}
\subsection{Contact}
\subsection{Mesh Tying and Multi-Point Constraints}

\section{Nonlinear Material Behavior}

\section{Solution Remapping}
\subsection{Severe Plastic Deformations}
\subsection{Mesh Tangling and Inversion}

\section{Fracture and Topology Change}

\section{Accuracy, Efficiency, and Stability}
\subsection{Adaptive Refinement}
\subsubsection{$h$-, $p$-, and $hp$-Adaptivity}

\chapter{Locking Phenomena in Computational Solid Mechanics}
%
\section{Manifestations of Locking}

\section{A Mathematical Treatment of the Locking Problem}

\section{Mitigation Techniques}
\subsection{Enhanced/Assumed Strain Methods and the Method of Incompatible Modes}
\subsection{Strain Projection Methods}
\subsubsection{B-Bar Method for Linear Problems}
\subsubsection{F-Bar Method for Nonlinear Kinematics}
\subsection{Stabilization Techniques}
\subsubsection{Hourglass Control}
\subsubsection{VEM Stabilization}
\subsection{Selective $p$-Refinement}

\chapter{The Partitioned Element Method}
%
\section{Overview}
\subsection{Justification}
\subsection{Advantages and Disadvantages}

\section{Abstract Framework}
\subsection{Construction of Element Appoximants}

\section{Specific Formulations}
\subsection{The Weak-Galerkin PEM}
\subsection{The Discontinuous-Galerkin PEM}
\subsection{The Petrov-Galerkin PEM}

\section{Mixed Discretizations}
\subsection{Element-Local Enhancement Functions}

\section{Implemenation of the PEM}
\subsection{Numerical Conditioning Issues}

\section{Numerical Evaluation of Performance}
\subsection{Convergence of the PEM}
\subsection{Parameter Sensitivity Analysis}
\subsection{Computational Efficiency}
\subsection{Resistance to Locking}

\chapter{Conclusions}
%
Evaluate the efficacy of the proposed formulations within the presented context. Assess whether there really is a silver bullet to the locking problem, or if we can expect for locking to be an inherrent issue with any numerical approximation method.


\bibliography{thesis_bibliography}

% The UMI abstract uses square brackets!
\UMIabstract[The abstract that is submitted to UMI must be formatted as shown in the example here. The body of the abstract cannot exceed 350 words. It should be in typewritten form, double-spaced, and on bond paper. It is important to write an abstract that gives a clear description of the content and major divisions of the dissertation, since UMI will publish the abstract exactly as submitted. Students completing their requirements under Plan A should provide extra copies of the typed summary for use by the dissertation committee during the examination.

The abstract that is submitted to UMI must be formatted as shown in the example here. The body of the abstract cannot exceed 350 words. It should be in typewritten form, double-spaced, and on bond paper. It is important to write an abstract that gives a clear description of the content and major divisions of the dissertation, since UMI will publish the abstract exactly as submitted. Students completing their requirements under Plan A should provide extra copies of the typed summary for use by the dissertation committee during the examination.

The abstract that is submitted to UMI must be formatted as shown in the example here. The body of the abstract cannot exceed 350 words. It should be in typewritten form, double-spaced, and on bond paper. It is important to write an abstract that gives a clear description of the content and major divisions of the dissertation, since UMI will publish the abstract exactly as submitted. Students completing their requirements under Plan A should provide extra copies of the typed summary for use by the dissertation committee during the examination.

The abstract that is submitted to UMI must be formatted as shown in the example here. The body of the abstract cannot exceed 350 words. It should be in typewritten form, double-spaced, and on bond paper. It is important to write an abstract that gives a clear description of the content and major divisions of the dissertation, since UMI will publish the abstract exactly as submitted. Students completing their requirements under Plan A should provide extra copies of the typed summary for use by the dissertation committee during the examination.]

\end{document} 
