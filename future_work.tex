\chapter{Conclusions and Future Work} \label{ch:future_work}
%
Evaluate the efficacy of the proposed formulations within the presented context. Assess whether there really is a silver bullet to the locking problem, or if we can expect for locking to be an inherent issue with any numerical approximation method.

Describe the nature of future work that will be done in this area of research.

Although some initial sensitivity analyses have been conducted for the DG-PEM with regard to the choice of penalty parameters $\alpha_{\sigma0}$, $\alpha_{\sigma1}$, a more thorough study is warranted. Particularly for the case where $\epsilon = +1$, $\alpha_{\sigma0} = \alpha_{\sigma1} = 0$ (corresponding to the OBB method \cite{Oden:98}), such settings may yield integration consistency without having to apply a gradient correction scheme. The OBB method is provably stable for approximation spaces consisting of polynomials of order $k=2$. This would present a challenge with regard to accurate and efficient numerical integration of quadratic polynomials.

It currently remains to be rigorously proven that the PEM yields a stable integration of the weak form for elements employing an edge-based partition and composite mid-point quadrature. Moreover, the effect of applying a gradient correction scheme to high-order elements needs to be investigated further.

An exploration of the PEM for explicit dynamics applications is also warranted. Given the preliminary results that were obtained for the Taylor bar problem run using implicit dynamics, the major challenges for explicit dynamics would likely come in the form of mesh tangling and element inversion. The stable time step size for explicit analyses may present an additional challenge, being limited typically by the smallest dimension of any element contained in the mesh. Consequently, the inclusion of elements with short edges may negatively impact the efficiency of the computations for explicit time-stepping if Voronoi meshes are utilized. However, considering the reasonably well-conditioned element stiffness matrices produced by the DG-PEM for elements with short edges, it may be possible to overcome some of these limitations. Further investigation would be required to confirm if this is the case. 

Another important consideration for the PEM in the context of non-linear solid mechanics applications is the issue of contact constraint enforcement. Particularly in 3D, the representation of a given element's shape functions on one of its faces may be discontinuous if the DG-PEM is employed. Consequently, it is not clear how best to enforce contact constraints upon such (discontinuous) deformed surfaces. Moreover, if the intermediary information regarding the representation of a given element face's shape functions is ultimately discarded (as suggested in \ref{ch:implementation}), then there would be no clear way to evaluate gap functions. The use of the PEM for contact applications would therefore necessitate storage of each interface element's geometric partition. Additionally, these elements' shape functions would need to be represented explicitly as piecewise polynomial fields.

The use of a node-on-face method of contact enforcement would likely present a number of challenges for the DG-PEM, owing to the inherently discontinuous representation of the elements' shape functions. The CG-PEM may be preferred in this setting, if only for the sake of defining shape functions for interface elements. Nonetheless, a face-on-face method would likely provide a better path forward. Specifically, the mortar-based approach proposed in \cite{Wopschall:17} makes use of a projected medial plane between each pair of contacting faces, enforcing the zero gap condition in a weak sense between each pair of overlapping faces. The use of such projections may allow for the DG-PEM to be utilized directly in this setting, even if the geometric representation for each face is discontinuous.

Ongoing efforts are currently directed at improving PEM elements' resistance to traditional forms of locking, including volumetric locking, and shear locking for thin elements. Numerous attempts made to improve the bending behavior of thin elements, but they encountered various issues. An examination of enhanced strain formulations in the context of PEM may yield fruitful results. In particular, because the PEM establishes a local approximation space to construct an element shape functions, this same space may be utilized to construct a set of enhancement functions defined locally on each element.

Another issue that was given considerable attention was the subject of non-planar faces and non-linear edges. It suffices to say that the preservation of polynomial completeness is not a trivial matter. Various approaches to address this issue were explored, though they all invariably encountered issues of numerical precision for nearly-planar faces and nearly-linear edges. Most of the aforementioned investigations examined only cases where the faces/edges were represented as piecewise linear manifolds. The consideration of curved geometries (such as NURBS surfaces) would likely present a greater challenge.

Extending the DG-PEM approach to higher-order elements (beyond $k=2$) poses a number of challenges, particularly with regard to efficiency. For successively higher order elements, the size of the local DG-PEM problems arising from (\ref{eq:dgpem_linear_system}) can become large. The assembly and solution of these systems of equations will therefore impart a large computational expense for analyses with high-order elements. In such cases, solving (\ref{eq:dgpem_linear_system}) directly may be impractical. As a possible alternative, it may be sufficient to solve (\ref{eq:dgpem_linear_system}) using an iterative procedure such as the conjugate gradient method. Certain essential characteristics of the resulting solution would need to be preserved, namely: polynomial completeness, and rank sufficiency of the resulting element stiffness matrices. If successful, such an approach could greatly reduce the computational burden of PEM elements, generally.

However, the performance of any iterative solver typically depends upon the numerical conditioning of the associated linear system of equations. Poor conditioning of the DG-PEM equation systems would therefore present a major obstacle to this approach. Even for reasonably well-scaled polynomial bases, preconditioning would likely be required to address this issue. Given the structure of the PEM, a geometric multigrid-based preconditioner could be readily obtained, though a simple Jacobi (diagonal) preconditioner might suffice. Regardless, a more thorough investigation would need to be carried out to determine the efficacy of any iterative approach.