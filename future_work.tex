\chapter{Conclusions and Future Work} \label{ch:future_work}
%
Based upon the preceding investigations, the DG-PEM performs well in comparison to the classical FEM for a variety of large deformation problems. Though the DG-PEM does not directly address the issue of locking, it nonetheless provides an avenue for exploring more general discretizations which may be less sensitive to locking. Additionally, the DG-PEM demonstrates crucial improvements over the CG-PEM and the VETFEM with regard to its ability to tolerate non-convex elements, and elements with comparatively short edges.

Although some initial sensitivity analyses have been conducted for the DG-PEM with regard to the choice of penalty parameters $\alpha_{\sigma0}$, $\alpha_{\sigma1}$, a more thorough study is warranted. Particularly for the case where $\epsilon = +1$, $\alpha_{\sigma0} = \alpha_{\sigma1} = 0$ (corresponding to the OBB method \cite{Oden:98}), such settings may yield integration consistency without having to apply a gradient correction scheme. Additionally, the OBB method is provably stable for approximation spaces consisting of polynomials of order $k=2$, though this would present a challenge with regard to accurate and efficient numerical integration of quadratic polynomials.

It currently remains to be rigorously proven that the PEM yields a stable integration of the weak form for elements employing an edge-based partition and composite mid-point quadrature. Moreover, the effects of using a gradient correction scheme on high-order elements needs to be investigated further.

An exploration of the PEM for explicit dynamics applications is also warranted. Given the preliminary results that were obtained for the Taylor bar problem (using implicit dynamics), the major challenges would likely be mesh tangling and element inversion. The stable time step size for explicit analyses may present an additional challenge, being limited by the smallest dimension of any element contained in the mesh. Consequently, the inclusion of elements with short edges may negatively impact the efficiency of the computations for explicit time-stepping if random Voronoi meshes are utilized. However, considering the reasonably well-conditioned element stiffness matrices produced by the DG-PEM for elements with short edges, it may be possible to overcome some of these limitations. Further investigation is required to confirm if this is the case.

Another important consideration for the PEM in the context of non-linear solid mechanics applications is the issue of contact constraint enforcement. Particularly in 3-dimensions, the representation of a given face's shape functions (and its deformed shape) will be discontinuous if the DG-PEM is employed. Consequently, it is not clear how best to enforce contact constraints upon these discontinuous surfaces. Moreover, if the intermediary information regarding the representation of a given element face's shape functions is ultimately discarded prior to the beginning of the analysis (as suggested in \ref{ch:implementation}), then there would be no clear way to evaluate gap functions. The use of the PEM for contact applications would therefore necessitate storage of each boundary face's geometric partition. Additionally, the shape functions for these faces would need to be represented explicitly as piecewise polynomial fields.

A node-on-face method of contact enforcement would likely present a number of challenges for the DG-PEM, owing to the inherently discontinuous representation of the deformed mesh boundary. The CG-PEM may be preferred in this setting, if only for the sake of defining shape functions on element faces. In contrast, a face-on-face method may be capable of handling contact between discontinuous surfaces. Specifically, the mortar-based approach proposed in \cite{Wopschall:17} constructs a projected medial plane between each pair of contacting faces, enforcing a zero-mean gap condition between each overlapping face pair. The use of such projections may allow for the DG-PEM to be utilized directly, even if the geometric representation for each face is discontinuous.

Ongoing efforts are currently directed at improving the resistance of PEM elements to traditional forms of locking, including volumetric locking, and shear locking for thin elements. Numerous attempts were made to try to improve the bending behavior of thin elements, in particular, but these encountered various issues. An examination of enhanced strain formulations in the context of PEM may yield fruitful results. In particular, because the PEM establishes a local approximation space to construct an element's shape functions, this same space could be utilized to construct a set of enhancement functions, specific to each element.

Another issue that was given considerable attention was the subject of non-planar faces and non-linear edges. Within this setting, it suffices to say that the preservation of polynomial completeness is not a trivial matter. Various approaches to address this issue were explored, though they all invariably encountered issues of numerical precision for nearly-planar faces and nearly-linear edges, resulting in very poor interpolation errors. Most of the aforementioned investigations examined only cases where the faces/edges were represented as piecewise linear manifolds. The consideration of curved geometries (such as NURBS surfaces) would present an additional challenge.

Extending the DG-PEM approach to higher-order elements (beyond $k=2$) poses a number of practical barriers, particularly with regard to efficiency. For successively higher order elements, the size of the local DG-PEM problems arising from (\ref{eq:dgpem_linear_system}) can become large. The assembly and solution of these systems of equations will therefore impart a large computational expense for analyses with high-order elements. In such cases, solving (\ref{eq:dgpem_linear_system}) directly may be impractical. As a possible alternative, it may be sufficient to solve (\ref{eq:dgpem_linear_system}) using an iterative approach, such as the conjugate gradient method. Certain essential characteristics of the resulting solution would need to be preserved, namely: polynomial completeness, and rank sufficiency of the resulting element stiffness matrices. If successful, such an approach could greatly reduce the computational burden of the PEM shape function construction procedure.

However, the performance of any iterative solver typically depends upon the numerical conditioning of the associated linear system of equations. Poor conditioning of the DG-PEM equation systems would therefore present a major obstacle to the aforementioned approach. Even for reasonably well-scaled polynomial bases, preconditioning would likely be required to address this issue. Given the structure of the PEM, a geometric multigrid-based preconditioner could be readily obtained, though a simple Jacobi (diagonal) preconditioner might suffice. A more thorough investigation would need to be carried out to determine the efficacy of any iterative approach for constructing PEM shape functions.

Finally, an investigation into the computational efficiency of the PEM still needs to be carried out. The computational expense of the PEM shape function construction procedure should be quantified in this setting. For general polyhedral discretizations, it is expected (and informally observed for the problems considered in this work) that the classical FEM outperforms the DG-PEM on meshes with comparable numbers of elements. Two key observations may explain this behavior: in comparison to polyhedral meshes with the same number of elements, hexahedral meshes typically possess fewer nodes (degrees of freedom), and the bandwidth of the resulting global stiffness matrix tends to be smaller, as well. Moreover, elements with arbitrary topology necessitate the use of data structures with variable size, incurring an additional overhead. The FEM is therefore expected to yield better computational efficiency, in general. Nonetheless, the PEM could potentially enable the use of novel linear solution methodologies (such as geometric multigrid methods), exploiting the geometric flexibility of the elements. Such approaches could help to improve the overall efficiency of the PEM.