\documentclass[12pt]{article}
\usepackage{amsmath}
\usepackage{amsfonts}
\usepackage{amssymb}

\title{Rotating annulus problem}
\author{}
\date{}

\begin{document}
\maketitle

Consider an annulus whose inner radius at $r = R_i$ is fixed, and whose outer radius $r = R_o$ rigidly rotates through a specified total angle. The displacement field for this motion is described by
\begin{equation}
	u_r = u_z = 0, \quad u_\theta = r \, \phi(r) \quad u_\theta (r = R_i) = 0, \, \, u_\theta (r = R_o) = R_o \, \phi (R_o),
\end{equation}
or alternatively as
\begin{equation}
	u_1 = r \left[ \cos \left(\theta + \phi \right) - \cos \theta \right] = -2 r \sin (\theta + \phi / 2) \sin (\phi / 2),
\end{equation}
\begin{equation}
	u_2 = r \left[ \sin \left(\theta + \phi \right) - \sin \theta \right] = 2 r \cos (\theta + \phi / 2) \sin (\phi / 2),
\end{equation}
based upon a coordinate transformation of the form:
\begin{equation}
	x_1 = r \cos \theta, \qquad x_2 = r \sin \theta,
\end{equation}
equivalently:
\begin{equation}
	r = \sqrt{x_1^2 + x_2^2}, \qquad \tan \theta = \frac{x_2}{x_1}.
	\label{inverse}
\end{equation}

If we suppose that $f (r, \theta)$ represents an arbitrary function of $r$ and $\theta$, then we may invoke the chain rule of differentiation to obtain expressions for $f_{,1}$ and $f_{,2}$, i.e.
\begin{equation}
	f_{,1} = f_{,r} r_{,1} + f_{,\theta} \theta_{,1},
	\label{partial1}
\end{equation}
\begin{equation}
	f_{,2} = f_{,r} r_{,2} + f_{,\theta} \theta_{,2}.
	\label{partial2}
\end{equation}
We proceed in writing out the expressions for $r_{,i}$ and $\theta_{,i}$ by differentiating the relationships in (\ref{inverse}) (once again employing the chain rule):
\begin{equation}
	r_{,1} = \left[ \sqrt{x_1^2 + x_2^2} \right]_{,1} = \frac{2 x_1}{2 \sqrt{x_1^2 + x_2^2}} = \frac{r \cos \theta}{r} = \cos \theta,
\end{equation}
\begin{equation}
	r_{,2} = \left[ \sqrt{x_1^2 + x_2^2} \right]_{,2} = \frac{2 x_2}{2 \sqrt{x_1^2 + x_2^2}} = \frac{r \sin \theta}{r} = \sin \theta,
\end{equation}
\begin{equation}
	\left[ \tan \theta \right]_{,i} = \left[ \tan \theta \right]_{,\theta} \theta_{,i} = \left[\sec^2 \theta\right] \theta_{,i} = \frac{\theta_{,i}}{\cos^2 \theta},
\end{equation}
\begin{equation}
	\frac{\theta_{,1}}{\cos^2 \theta} = \left[ \tan \theta \right]_{,1} = \left[ \frac{x_2}{x_1} \right]_{,1} = - \frac{x_2}{x_1^2},
\end{equation}
\begin{equation}
	\theta_{,1} = - \frac{\cos^2 \theta \, r \sin \theta}{r^2 \cos^2 \theta} = - \frac{\sin \theta}{r},
\end{equation}
\begin{equation}
	\frac{\theta_{,2}}{\cos^2 \theta} = \left[ \tan \theta \right]_{,2} = \left[ \frac{x_2}{x_1} \right]_{,2} = \frac{1}{x_1}
\end{equation}
\begin{equation}
	\theta_{,2} = \frac{\cos^2 \theta}{r \cos \theta} = \frac{\cos \theta}{r}.
\end{equation}
Given these expressions for $r_{,i}$ and $\theta_{,i}$, we may now rewrite equations (\ref{partial1}) and (\ref{partial2}) as:
\begin{equation}
	f_{,1} = f_{,r} \cos \theta - f_{,\theta} \frac{\sin \theta}{r},
\end{equation}
\begin{equation}
	f_{,2} = f_{,r} \sin \theta + f_{,\theta} \frac{\cos \theta}{r}.
\end{equation}

After some algebraic manipulation, we may write out expressions for $u_{\alpha,\beta} \, \forall \alpha, \beta = 1, 2$:
\begin{eqnarray}
	u_{1,1} = - 1 + \cos \phi - r \phi_{,r} \cos \theta \sin (\theta + \phi),
\end{eqnarray}
\begin{eqnarray}
	u_{1,2} = - \sin \phi - r \phi_{,r} \sin \theta \sin (\theta + \phi),
\end{eqnarray}
\begin{eqnarray}
	u_{2,1} = + \sin \phi + r \phi_{,r} \cos \theta \cos (\theta + \phi),
\end{eqnarray}
\begin{eqnarray}
	u_{2,2} = - 1 + \cos \phi + r \phi_{,r} \sin \theta \cos (\theta + \phi).
\end{eqnarray}
If we define the following terms:
\begin{equation}
	\theta' = \theta + \phi,
\end{equation}
\begin{equation}
	\mathbf{R}_\phi = \left[ \begin{array}{ccc} 
	\cos \phi & - \sin \phi & 0 \\ 
	\sin \phi & \cos \phi & 0 \\ 0 & 0 & 1 \end{array} \right],
\end{equation}
\begin{equation}
	\mathbf{e}_{\theta'} = \left\{ \begin{array}{c} 
	- \sin \theta' \\ \cos \theta' \\ 0 \end{array} \right\}, \quad
	\mathbf{e}_r = \left\{ \begin{array}{c} 
	\cos \theta \\ \sin \theta \\ 0 \end{array} \right\},
\end{equation}
\begin{equation}
	\mathbf{W}_\phi = \mathbf{e}_{\theta'} \otimes \mathbf{e}_r ,
\end{equation}
then we may express the deformation gradient $\mathbf{F}$ as
\begin{equation}
	\mathbf{F} = \frac{\partial \mathbf{x}}{\partial \mathbf{X}} = \mathbf{R}_\phi + r \phi_{,r} \mathbf{W}_\phi.
\end{equation}
Using the matrix determinant lemma, it is easy to verify that this deformation is volume preserving (i.e. $\det (\mathbf{F}) = 1$.)

The left Cauchy-Green deformation tensor is then
\begin{equation}
	\mathbf{B} = \mathbf{F} \mathbf{F}^T = \mathbf{1} + r \phi_{,r} (\mathbf{R}_\phi \mathbf{W}_\phi^T + \mathbf{W}_\phi \mathbf{R}_\phi^T) + r^2 \phi_{,r}^2 \mathbf{e}_{\theta'} \otimes \mathbf{e}_{\theta'},
\end{equation}
or
\begin{equation}
	\mathbf{B} = \mathbf{1} +  r \phi_{,r} (\mathbf{e}_{r'} \otimes \mathbf{e}_{\theta'} + \mathbf{e}_{\theta'} \otimes \mathbf{e}_{r'}) + r^2 \phi_{,r}^2 \mathbf{e}_{\theta'} \otimes \mathbf{e}_{\theta'},
\end{equation}
where $\mathbf{e}_{r'} = \left\{ \begin{array}{ccc} \cos \theta' & \sin \theta' & 0 \end{array} \right\}^T$. Further,
\begin{equation}
	\mathbf{e}_{r'} \otimes \mathbf{e}_{\theta'} + \mathbf{e}_{\theta'} \otimes \mathbf{e}_{r'} = \left[ \begin{array}{ccc} 
	- \sin 2 \theta' & \cos 2 \theta' & 0 \\ 
	\cos 2 \theta' & \sin 2 \theta' & 0 \\ 0 & 0 & 0 \end{array} \right],
\end{equation}
and
\begin{equation}
	\mathbf{e}_{r'} \otimes \mathbf{e}_{r'} = \frac{1}{2} \left[ \begin{array}{ccc} 
	1 + \cos 2 \theta' & \sin 2 \theta' & 0 \\ 
	\sin 2 \theta' & 1 - \cos 2 \theta' & 0 \\ 0 & 0 & 0 \end{array} \right],
\end{equation}
yielding
\begin{equation}
	\mathbf{B}_2 = \mathbf{1}_2 + r \phi_{,r} \left[ \begin{array}{cc} 
	- \sin 2 \theta' & \cos 2 \theta' \\ 
	\cos 2 \theta' & \sin 2 \theta' \end{array} \right] + \frac{r^2 \phi_{,r}^2}{2} \left[ \begin{array}{ccc} 
	1 + \cos 2 \theta' & \sin 2 \theta' \\ 
	\sin 2 \theta' & 1 - \cos 2 \theta' \end{array} \right],
\end{equation}
where $\mathbf{B}_2 = \mathbf{B} - \mathbf{e}_z \otimes \mathbf{e}_z$. It can be shown that the eigenvalues of $\mathbf{B}_2$ are
\begin{equation}
	\lambda_1 = - \frac{1}{2} \left[ - r^2 \phi_{,r}^2 + \sqrt{r^4 \phi_{,r}^4 + 4 r^2 \phi_{,r}^2} - 2 \right],
\end{equation}
\begin{equation}
	\lambda_2 = + \frac{1}{2} \left[ r^2 \phi_{,r}^2 + \sqrt{r^4 \phi_{,r}^4 + 4 r^2 \phi_{,r}^2} + 2 \right],
\end{equation}
with corresponding eigenvectors
\begin{equation}
	\mathbf{v}_1 = \left\{ \begin{array}{cc} \frac{-\sqrt{r^4 \phi_{,r}^4 + 4 r^2 \phi_{,r}^2} + r^2 \phi_{,r}^2 \cos 2 \theta' - 2 r \phi_{,r} \sin 2 \theta'}{2 r \phi_{,r} \cos 2 \theta' + r^2 \phi_{,r}^2 \sin 2 \theta'} & 1 \end{array} \right\},
\end{equation}
\begin{equation}
	\mathbf{v}_2 = \left\{ \begin{array}{cc} \frac{\sqrt{r^4 \phi_{,r}^4 + 4 r^2 \phi_{,r}^2} + r^2 \phi_{,r}^2 \cos 2 \theta' - 2 r \phi_{,r} \sin 2 \theta'}{2 r \phi_{,r} \cos 2 \theta' + r^2 \phi_{,r}^2 \sin 2 \theta'} & 1 \end{array} \right\}.
\end{equation}
\begin{equation}
	|| \mathbf{v}_1 || = \frac{\sqrt{2 (x^2 + 4 + \sqrt{x^2+4} (2 \sin y - x \cos y))}}{x \sin y + 2 \cos y}
\end{equation}
\begin{equation}
	|| \mathbf{v}_2 || = \frac{\sqrt{2 (x^2 + 4 - \sqrt{x^2+4} (2 \sin y - x \cos y))}}{x \sin y + 2 \cos y}
\end{equation}
\begin{equation}
	\mathbf{v}_1 = \frac{1}{\sqrt{- 2 \sqrt{x^2+4} (x \cos y - 2 \sin y - \sqrt{x^2+4})}} \left\{ \begin{array}{c} x \cos y - 2 \sin y - \sqrt{x^2 + 4} \\ 2 \cos y + x \sin y \end{array} \right\},
\end{equation}
\begin{equation}
	\mathbf{v}_2 = \frac{1}{\sqrt{2 \sqrt{x^2+4} (x \cos y - 2 \sin y + \sqrt{x^2+4})}} \left\{ \begin{array}{c} x \cos y - 2 \sin y + \sqrt{x^2 + 4} \\ 2 \cos y + x \sin y \end{array} \right\}.
\end{equation}
\begin{equation}
	\lambda_1 = \frac{\sqrt{x^2 + 4}}{2} \left[ \sqrt{x^2 + 4} - x \right],
\end{equation}
\begin{equation}
	\lambda_2 = \frac{\sqrt{x^2 + 4}}{2} \left[ \sqrt{x^2 + 4} + x \right],
\end{equation}
Consequently,
\begin{equation}
	\mathbf{B}_2 = \lambda_1 \mathbf{v}_1 \otimes \mathbf{v}_1 + \lambda_2 \mathbf{v}_2 \otimes \mathbf{v}_2,
\end{equation}
and the in-plane Hencky strain is
\begin{equation}
	\mathbf{h}_2 = \frac{1}{2} \left[ \ln (\lambda_1) \mathbf{v}_1 \otimes \mathbf{v}_1 + \ln (\lambda_2) \mathbf{v}_2 \otimes \mathbf{v}_2 \right],
\end{equation}
where
\begin{eqnarray}
	\mathbf{v}_1 \otimes \mathbf{v}_1 = \frac{1}{- 2 \sqrt{x^2+4} (x \cos y - 2 \sin y - \sqrt{x^2+4})} \\
	 \left[ \begin{array}{cc} (x \cos y - 2 \sin y - \sqrt{x^2 + 4})^2 &
	 (x \cos y - 2 \sin y - \sqrt{x^2 + 4})(2 \cos y + x \sin y) \\ 
	 (x \cos y - 2 \sin y - \sqrt{x^2 + 4})(2 \cos y + x \sin y) &
	 4 \cos^2 y + x^2 \sin^2 y + 4 x \cos y \sin y \end{array} \right]
\end{eqnarray}

For this special motion, it can be shown that the Hencky model of elasticity results in the following expressions for the in-plane principal stresses:
\begin{equation}
	\sigma_1 = \mu \ln (\lambda_1), \quad \sigma_2 = \mu \ln (\lambda_2),
\end{equation}
and it suffices to show that
\begin{equation}
	\sigma_{1,r} (\cos \theta' + \sin \theta') + \frac{\sigma_{1,\theta'}}{r} (\cos \theta' - \sin \theta') = 0,
\end{equation}
\begin{equation}
	\sigma_{2,r} (\cos \theta' + \sin \theta') + \frac{\sigma_{2,\theta'}}{r} (\cos \theta' - \sin \theta') = 0.
\end{equation}
Since the above equations hold irrespective of which value is chosen for $\theta$, we may examine the trival case of $\theta = 0$, and thus
\begin{equation}
	\sigma_{1,r} (\cos \phi + \sin \phi) + \frac{\sigma_{1,\phi}}{r} (\cos \phi - \sin \phi) = 0,
\end{equation}
\begin{equation}
	\sigma_{2,r} (\cos \phi + \sin \phi) + \frac{\sigma_{2,\phi}}{r} (\cos \phi - \sin \phi) = 0,
\end{equation}

Consider the stress tensor:
\begin{equation}
  \boldsymbol{\sigma} = \sigma_{rr} \mathbf{e}_r \otimes \mathbf{e}_r + \sigma_{\theta \theta} \mathbf{e}_\theta \otimes \mathbf{e}_\theta + \sigma_{zz} \mathbf{e}_z \otimes \mathbf{e}_z + \frac{1}{2} \left[ \sigma_{r\theta} (\mathbf{e}_r \otimes \mathbf{e}_\theta + \mathbf{e}_\theta \otimes \mathbf{e}_r) + \sigma_{rz} (\mathbf{e}_r \otimes \mathbf{e}_z + \mathbf{e}_z \otimes \mathbf{e}_r) + \sigma_{z\theta} (\mathbf{e}_z \otimes \mathbf{e}_\theta + \mathbf{e}_\theta \otimes \mathbf{e}_z),
\end{equation}
and it's divergence:
\begin{equation}
  \nabla \cdot \boldsymbol{\sigma} = \sigma_{rr} \mathbf{e}_r \otimes \mathbf{e}_r + \sigma_{\theta \theta} \mathbf{e}_\theta \otimes \mathbf{e}_\theta + \sigma_{zz} \mathbf{e}_z \otimes \mathbf{e}_z + \frac{1}{2} \left[ \sigma_{r\theta} (\mathbf{e}_r \otimes \mathbf{e}_\theta + \mathbf{e}_\theta \otimes \mathbf{e}_r) + \sigma_{rz} (\mathbf{e}_r \otimes \mathbf{e}_z + \mathbf{e}_z \otimes \mathbf{e}_r) + \sigma_{z\theta} (\mathbf{e}_z \otimes \mathbf{e}_\theta + \mathbf{e}_\theta \otimes \mathbf{e}_z),
\end{equation}

\end{document}
