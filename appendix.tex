\chapter{Exact Solution for the Incompressible Twisting Annulus Problem}

An exact solution for the twisting annulus problem described in chapter \ref{ch:results} may be obtained by considering the stress divergence equations in cylindrical polar coordinates:
\begin{equation}
  \nabla \cdot \boldsymbol{\sigma} = \left\{ \begin{array}{c} \frac{\partial \sigma_{rr}}{\partial r} + \frac{\sigma_{rr}}{r} + \frac{1}{r} \frac{\partial \sigma_{\theta r}}{\partial \theta} + \frac{\partial \sigma_{z r}}{\partial z} - \frac{\sigma_{\theta \theta}}{r} \\
    \frac{1}{r} \frac{\partial \sigma_{\theta \theta}}{\partial \theta} + \frac{\partial \sigma_{r\theta}}{\partial r} + \frac{\sigma_{r\theta}}{r} + \frac{\sigma_{\theta r}}{r} + \frac{\partial \sigma_{z \theta}}{\partial z} \\
    \frac{\partial \sigma_{z z}}{\partial z} + \frac{\partial \sigma_{r z}}{\partial r} + \frac{\sigma_{r z}}{r} + \frac{1}{r} \frac{\partial \sigma_{\theta z}}{\partial \theta} \end{array} \right\} = \mathbf{0}.
\end{equation}
By the assumptions of plane strain and axisymmetry, it is rationalized that $\boldsymbol{\sigma} (r)$ must be a function of $r$, alone. Therefore:
\begin{equation}
  \nabla \cdot \boldsymbol{\sigma} = \left\{ \begin{array}{c} \frac{\partial \sigma_{rr}}{\partial r} + \frac{\sigma_{rr} - \sigma_{\theta \theta}}{r} \\
    \frac{\partial \sigma_{r\theta}}{\partial r} + \frac{2 \sigma_{r\theta}}{r} \\
    \frac{\partial \sigma_{r z}}{\partial r} + \frac{\sigma_{r z}}{r} \end{array} \right\} = \mathbf{0}.
\end{equation}
Furthermore, by the assumptions of plane strain, it is observed that $\sigma_{rz} = 0$ and $\sigma_{\theta z} = 0$. Additionally, imposing the incompressibility condition $\nabla \cdot \mathbf{v} = \text{tr} (\mathbf{D}) = 0$:
\begin{equation}
  \text{tr} (\dot{\boldsymbol{\sigma}}) = 0 \, \Rightarrow \, \text{tr} (\boldsymbol{\sigma}) = 0 \, \forall t.
\end{equation}
Suppose that $p = \sigma_{zz} = 0$, and therefore $\sigma_{\theta \theta} = - \sigma_{rr}$. This yields 2 governing differential equations for $\sigma_{rr}$ and $\sigma_{r \theta}$:
\begin{equation}
  \frac{\partial \sigma_{rr}}{\partial r} + \frac{2 \sigma_{rr}}{r} = 0, \quad \frac{\partial \sigma_{r\theta}}{\partial r} + \frac{2 \sigma_{r\theta}}{r} = 0,
\end{equation}
whose solutions are of the form
\begin{equation}
  \sigma_{rr} = \frac{\sigma}{r^2}, \quad \sigma_{r\theta} = \frac{\tau}{r^2},
\end{equation}
where $\sigma(t)$ and $\tau(t)$ are independent functions of time. Consequently,
\begin{equation}
  \left\{ \begin{array}{c} \sigma_{rr} \\ \sigma_{\theta \theta} \\ \sigma_{r \theta} \end{array} \right\} = r^{-2} \left\{ \begin{array}{c} +\sigma \\ -\sigma \\ \tau \end{array} \right\}.
\end{equation}

Under the assumption of incompressibility, the velocity field is $v_r = 0$, $v_z = 0$, and $v_\theta (r) = r \dot{\phi}(r)$, for some $\phi(r,t)$. The velocity gradient (in cylindrical polar coordinates) is written:
\begin{equation}
  \mathbf{L} = \nabla \mathbf{v} = \left[ \begin{array}{ccc} v_{r,r} & v_{r,\theta}/r-v_\theta /r & v_{r,z} \\
      v_{\theta,r} & v_{\theta,\theta}/r + v_r /r & v_{\theta,z} \\
      v_{z,r} & v_{z,\theta} /r & v_{z,z} \end{array} \right] = 
  \left[ \begin{array}{ccc} 0 & -\dot{\phi} & 0 \\
      \dot{\phi} + r \dot{\phi}_{,r} & 0 & 0 \\
      0 & 0 & 0 \end{array} \right].
\end{equation}
The corresponding rate of deformation and spin tensors are:
\begin{equation}
  \mathbf{D} = \frac{r \dot{\phi}_{,r}}{2} \left[ \begin{array}{ccc} 0 & +1 & 0 \\
      +1 & 0 & 0 \\
      0 & 0 & 0 \end{array} \right], \quad \mathbf{W} = 
  \frac{2 \dot{\phi} + r \dot{\phi}_{,r}}{2} \left[ \begin{array}{ccc} 0 & - 1 & 0 \\
      +1 & 0 & 0 \\
      0 & 0 & 0 \end{array} \right].
\end{equation}
The stress rate equations resulting from $\dot{\boldsymbol{\sigma}} = \mathbb{C} : \mathbf{D} + \mathbf{W} \boldsymbol{\sigma} - \boldsymbol{\sigma} \mathbf{W}$ are
\begin{equation}
  \left\{ \begin{array}{c} \dot{\sigma}_{rr} \\ \dot{\sigma}_{\theta \theta} \\ \dot{\sigma}_{r \theta} \end{array} \right\} = \left\{ \begin{array}{c} 0 \\ 0 \\ \mu r \dot{\phi}_{,r} \end{array} \right\} + 
  (2 \dot{\phi} + r \dot{\phi}_{,r}) \left\{ \begin{array}{c} - \sigma_{r \theta} \\ \sigma_{r \theta} \\ \sigma_{rr} \end{array} \right\},
\end{equation}
or (in terms of $\sigma$, $\tau$):
\begin{equation}
  \left\{ \begin{array}{c} \dot{\sigma} \\ \dot{\tau} \end{array} \right\} = \left\{ \begin{array}{c} 0 \\ \mu r^3 \dot{\phi}_{,r} \end{array} \right\} + (2 \dot{\phi} + r \dot{\phi}_{,r}) \left[ \begin{array}{cc} 0 & -1 \\ 1 & 0 \end{array} \right] \left\{ \begin{array}{c} \sigma \\ \tau \end{array} \right\},
\end{equation}
which must be valid for all $r, \, t$. Assume that $\dot{\phi}(r)$ is a function of $r$ (and not of $t$), corresponding to a steady rate of deformation. By recognizing that $\sigma_{,r} = \tau_{,r} = 0$, one obtains the condition
\begin{equation}
  3 \dot{\phi}_{,r} + r \dot{\phi}_{,rr} = 0,
\end{equation}
implying $\dot{\phi}_{,r} = B r^{-3}$, and $\phi (r, t) = (A - B r^{-2} / 2)t$. Thus
\begin{equation}
  \left\{ \begin{array}{c} \dot{\sigma} \\ \dot{\tau} \end{array} \right\} = \left\{ \begin{array}{c} 0 \\ \mu B \end{array} \right\} + 2 A \left[ \begin{array}{cc} 0 & -1 \\ 1 & 0 \end{array} \right] \left\{ \begin{array}{c} \sigma \\ \tau \end{array} \right\},
\end{equation}
and
\begin{equation}
  \sigma(t) = - \frac{\mu B}{2 A} - C_2 \sin (2 A t) + C_1 \cos (2 A t),
\end{equation}
\begin{equation}
  \tau(t) = C_1 \sin (2 A t) + C_2 \cos (2 A t).
\end{equation}
Imposing the initial conditions $\sigma(0) = \tau(0) = 0$ results in:
\begin{equation}
  \sigma(t) = \frac{\mu B}{2 A} \left[ \cos (2 A t) - 1 \right], \quad \tau(t) = \frac{\mu B}{2 A} \sin (2 A t).
\end{equation}
Imposing the boundary conditions $\phi(R_i,t) = 0 \, \, \forall t$, $\phi(R_o,t) = \Phi \, t \, \, \forall t$ yields:
\begin{equation}
  A = \frac{R_o^{2}}{R_o^{2} - R_i^{2}} \Phi, \quad B = 2 \frac{R_o^{2} R_i^{2}}{R_o^{2} - R_i^{2}} \Phi.
\end{equation}
The final analytical solutions for the displacement and stress fields are:
\begin{equation}
  u_r = u_z = 0, \quad u_\theta = \frac{R_o^2}{r} \frac{r^2 - R_i^2}{R_o^{2} - R_i^{2}} \Phi t,
\end{equation}
\begin{equation}
  \sigma_{rr} = - \sigma_{\theta \theta} = \mu \frac{R_i^{2}}{r^{2}} \left[ \cos \left( 2 \frac{R_o^{2}}{R_o^{2} - R_i^{2}} \Phi t \right) - 1 \right], \quad \sigma_{r \theta} = \mu \frac{R_i^{2}}{r^{2}} \sin \left( 2 \frac{R_o^{2}}{R_o^{2} - R_i^{2}} \Phi t \right).
\end{equation}
According to \cite{Brannon:11}, the above solution for $\sigma_{r \theta}$ is also valid for compressible elastic materials.