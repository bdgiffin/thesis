\chapter{Implementational Framework for the PEM}

% emphasize element-level computations as being tantamount to FE meshing & solution procedure
% must discretize the geometry efficiently, get accurate computations of geometric quantities
% computations dominated by systems of equations, require efficient linear solvers (CG, etc.)
% explore accuracy of computations and impact on global solution behavior

\section{Element Partitioning Methods}
	
	\subsection{Deterministic Partitioning} % for star-convex shapes
		% use edge-based partitioning scheme for simplicity
		% heuristically, should provide relatively stable quadrature rules
		% limited to star-convex shapes, unless able to sub-divide into convex regions
		% suggest leveraging existing meshing toolkits to this purpose
		% caveat: discretizations should remain relatively efficient
		% much work remains with regard to assessing the quality of a partition

	\subsection{Voronoi Partitioning} % for arbitrary domains
		% discuss options (e.g. using qhull) to carry out efficient voronoi partition
		% allude to algorithms (Ebeida) for doing this on arbitrary domains
		% may encounter degeneracies (short egdes) leading to ill-conditioning

\section{Graph-based Geometry}
	% invoke notions of generic programming paradigms
		% allows for more code reuse/abstraction
		% amenable to generic pointers, etc.
		% less favorable in some computing environments/languages (CUDA, Fortran)

	\subsection{Geometric Entities}
		% verticies, segments, facets, cells
		% show representative tree diagram
		% each d-entity shares 1 (d-1)-entity in common with another on a d-manifold
	
	\subsection{Element Manifolds}
		% an element may share multiple geometric entities in common with another element
			% such sets are classifiable as "combinatorial" manifolds
			% entities owned solely by a single element are likewise identified as such
		% nodes, edges, faces, elements
		% show representative tree diagram

\section{Construction of Shape Functions}

	\subsection{Construction of Partition-based Basis Functions}
		% must construct basis functions on the partitioned geometry
		% nature of basis functions requires knowledge of which approx. method used
		% add element-wide polynomial enhancements as needed
		% elaborate on choice of polynomial basis to avoid conditioning issues

	\subsection{Generic Manifold Solution Procedure}
		% solution methods are abstracted, assisted by graph-based generic programming

	\subsection{Entity-level Calculations and Assembly}
		% numerical integration on arbitrary polytopes
			% Lassere integration
			% Gauss, Dunavant quadrature on simplicies

	\subsection{Constraint Enforcement}
		% pose set of linear constraints
		% suggest Lagrange multiplier method
		% alternatively suggest pseudo-inverse method to handle constraints

	\subsection{Linear Solution Method}
		% the need for pseudo-inverse computations in the case of non-planar faces
			% discuss challenges, limitations

\section{Construction of Quadrature Rules}
	% remark that applied consistency constraints must acknowledge choice of quad. rule
	% degree of violation of consistency will depend in part on quad. rule accuracy
	
	\subsection{Composite Mid-point Quadrature}
		% discuss impact on quadrature error/accuracy
			% introduces need for PG approach to robustly preserve consistency

\section{Verification Testing}
	
	\subsection{Tests for Reproducibility}
		% verify pass/fail status of reproducibility
		% state effects of non-planar faces
		% should yield exact results at each point for polys up to specified order

	\subsection{Tests for Consistency}
		% verify pass/fail status of reproducibility
		% examine effects of different quadrature rules
		% should yield exact results for polys up to specified order (patch tests)

\section{Numerical Considerations}

	\subsection{Linear System Conditioning}
		% compute error bound on linear system condition number
		% show that it is poor for DG/WG methods, according to choice of poly. basis
		% show that it gets bad if there are degeneracies or bad entity size ratios

	\subsection{Investigation of Shape Function Accuracy}
		% demonstrate accuracy problems for high-order element formulations
		% demonstrate accuracy problems for degenerate features
		% demonstrate numerical problems for non-planar faces

	\subsection{Investigation of Quadrature Accuracy}
		% introduce measure of quadrature error
		% check robustness of quadrature error against element distortion (2012 paper)

