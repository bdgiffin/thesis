\chapter{A General Framework for Incremental Kinematics}

% OVERVIEW
	% state the problem:
		% hypoelastic corotational time integration
		% most material models take stretching as the only input
		% must decompose motion into stretching & rotation, and apply separately

\section{Generalized Materail Update}

	% material state must be described at a discrete point
		% kinematic state data, strain
		% material (internal) state data, stress
	% material state updated from state k to k+1
		% pass an increment of deformation (F_hat)
		% decompose F_hat into rotation and stretch increments
		% update material state in sequence (stretch, rotate)

\section{Kinematic Splitting}

	% choice of splitting determines which corotational rate
	% describe methods for performing kinematic splitting
		% Hughes-Winget
		% Rashid 1993
	% describe new method
		% exponential Lie splitting of the velocity gradient L
		% use method of Scherzinger & Dorhmann for eignedecomp
		% use accurate derivatives based on eigenstuff

\section{Exploration of Accuracy}

	% single step accuracy convergence rate comparisons
		% stretch & rotate problem
		% simple shear problem
	% twisting prism problem

\section{Exploration of Convergence}

	% pressurized eccentric cylinder problem
