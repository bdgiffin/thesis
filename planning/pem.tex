\chapter{Partitioned Element Methods}

% OVERVIEW
	% goal is to construct shape functions on arbitrary polytopes
	% want efficient and stable quadrature rules
	% depart from the idea of shape functions defined pointwise
	% invoke idea behind FE approximation spaces: use discretization
	% solve local approximation problem to define SFs
	% enforce constraints to attain reproducibility and consistency

\section{The Element Partition}
	% consider an arbitrary element domain
	% discretize domain into conforming sub-domains: a partition
	% limitation: sub-domains must be polytopes
	% shape constraints imposed according to chosen approx. method

\section{Partition-Based Approximation Spaces}
	% define SFs as piece-wise polynomials on the partitioned geometry
	% try to minimize number of basis functions for efficiency

	\subsection{Finite Element Basis Functions}
		% unknowns stored at verticies only (efficient)
		% easy generalization to high order
		% limited to canonical shapes (tris/tets, quads/hexas)
		% quadrature rules much simpler (may use high order)
		% equivalent to Joe Bishop's approach (w/ Laplace SFs)
		% less sensitive to conditioning problems

	\subsection{Weak Galerkin Basis Functions}
		% unknowns stored on cells and edges
		% generalization to high order more expensive
		% Rashid and Sadri 2012 approach (low order)
		% may suffer from conditioning issues at high order

	\subsection{Discontinuous Galerkin Basis Functions}
		% unknowns stored on cells
		% easy generalization to high order
		% New/current approach
		% may suffer from conditioning issues at high order

	\subsection{Virtual Element Basis Functions}
		% unknowns stored at verticies only (efficient)
		% high order generalization effected via serendipity SFs
		% generalizes to arbitrary polytopes
		% New/speculated approach
		% less sensitive to conditioning problems

\section{Definition of Quadrature Rules}
	% quadrature rules are defined/linked to the chosen partition
	% 1-to-1 relation between partitioned cells and quadrature points
	% use Lassere integration to get weights of polytopal domains
	% employ low-order quadrature rules for the sake of efficiency
	% FE/simplicial discretization allows high-order composite rules

\section{Shape Function Boundary Value Problems}
	% setup: define a space of approximating functions over the element (basis)
	% goal:  select shape functions from this basis that are "optimal"
		% trial functions should reproduce polynomials as best they can
		% test functions should be close to trial functions (for stability)
		% test functions should satisfy compatibility requirements weakly
		% test functions should satisfy quad. consistency requirements against polys.
	% must propose a positive-definite functional to minimize on E
	% must enforce constraints on minimization
		% consistency enforced via Lagrange multipliers
	% develop a resulting system of equations
	% condense out all internal dofs in terms of element nodal values
	% allows for the inclusion of optional enhanced dofs, if needed
	% want SFs to be free of pathologies
	% Laplace shape functions
		% Joe Bishop's approach
		% Rashid and Sadri 2012 approach
	% Least-squares penalty method
		% necessary to stabilize WG/DG approaches
	% Other methods
		% many things to explore in this regard...
		% biharmonic? (for higher-order completeness? C^1 SFs are hard)

\section{Shape Function Constraints}

	\subsection{Stability}
		% driven by the inf-sup conditions
		% handled via sufficient quadrature arrangement

	\subsection{Compatibility}
		% driven by the generalized patch test
		% handled by constraints on approximation spaces/BVP

	\subsection{Consistency}
		% driven by Galerkin exactness
		% enforced by quadrature accuracy
		% alternatively via lagrange multiplier constraint

	\subsection{Reproducibility}
		% driven by completeness requirements
		% handled by BVP/functional specification
		% alternatively enforced via constraints

\section{Specific Formulations}
	% Harmonic SFs with an FE basis (Joe Bishop's approach)
	% Stabilized harmonic SFs with WG basis (Rashid and Sadri approach)
	% C^k penalty approximating SFs with DG basis (new method)
	% Harmonic SFs with VE basis + stabilization (new method)

\section{PEM Enrichment Functions}
	% GENERAL RULE: enhancements should seek to increase completeness
		% witnessed in most incompatible modes formulations
		% only way to improve accuracy: increase order, locally
	% balance stability, consistency, compatibility, reproducibility
		% bubble function construction closely tied to other SFs
		% enhancements must be orthogonal to low-order terms
		% must satisfy compatibility weakly over whole boundary
	% element-wide high order enhancement functions
		% inexpensive way to improve polynomial completeness
		% may apply to boundary dofs, or internal dofs

