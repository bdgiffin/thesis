\chapter{Numerical Investigations of the PEM}

	% compare efficacy of various PEM approaches
	% try to see which formulation is the most robust/accurate
	% investigate sensitivity to choice of internal penalty parameters in the element formulations
	% examine behavior of the elements in canonical locking type problems
		% volumetric locking
		% shear locking (thin elements)

	\section{Patch Tests}

		\subsection{Linear Patch Tests}
			% 2D & 3D
			% with and without consistency correction
			% BG vs PG
			% sensitivity to patch distorsion

		\subsection{Quadratic Patch Tests}
			% 2D & 3D
			% with and without consistency correction
			% BG vs PG
			% sensitivity to patch distorsion

	\section{Convergence Studies}
		% investigate convergence for high-order shape functions
		% exploration of accuracy/convergence in refinement limit (limited by SF errors?)
			% exploration of effects of numerical accuracy & system conditioning

		\subsection{Plate with a Circular Hole in Tension}
			% compare against serendipity and lagrange elements
			% demonstrate loss of convergence for serendipity isoparametric elements
			% demonstrate higher-order convergence for PEM shape functions

	\section{Sensitvity Analyses}
		% examine sensitivity to mesh distortions
		% sensitivity to internal (penalty) parameters
		% senstivity to material parameters / dependency
		% assess whether optimal values exist, whether formulation too sensitive

	\section{Locking Behavior}
		% primary goal is to assess performance for coarse resolution meshes

		\subsection{Twisting Annulus}
			% inner radius remains fixed, outer radius rigidly rotates
			% individual elements should experience simple shear AND some bending
				% bending deformation causes locking/bad stress field
			% nearly incompressible material model
			% measure stress error (should exhibit locking behavior)
			% there should exist an exact, axisymmetric solution
				% provide derivation/approximation to the analytic solution

		\subsection{Thin Plate in Bending}
			% use global error metrics to assess performance (L2 and H1 measures)
				% should have 
			% investigate sensitivity to mesh distortions
			% investigate sensitivity to material parameters

		\subsection{Pinched Pipe}
			% run using finite deformations
			% refer to Bischoff's work on continuum shell formulations
			% prefer a polyhedral mesh, if this is feasible
			% compare against standard conforming hex elements (expect poor performance)
			% displacement controlled pinch to illustrate localization performance

