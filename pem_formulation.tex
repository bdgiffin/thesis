\chapter{The Partitioned Element Method}
%
\section{Overview}
\subsection{Rationalization of the Proposed Approach}
\subsection{Advantages and Disadvantages}

\section{An Abstract Partitioned Element Framework}
\subsection{The Element Partition}
\subsection{Partition-Based Approximation Spaces}
\subsection{Selection of an Appropriate Objective Functional}

Once we have specified a given approximation space of functions, our goal is then to select the function from this space which best represents the nodal data that we are attempting to interpolate over the element. The essential question is this: by what metric should we objectively assess the appropriateness of a given approximating function?

\subsection{Construction of Element Appoximants}

\section{Essential Requirements of the PEM}
\subsection{Reproducibility}
\subsubsection{Requirements of the Approximation Space}
\subsection{Compatibility}
\subsubsection{Weak Enforcement of Continuity}
\subsection{Stability}
\subsubsection{Restrictions on the Element Partition}
\subsection{Variational Consistency}
\subsubsection{Weak Enforcement of Consistency}
\subsubsection{Bubnov-Galerkin and Petrov-Galerkin Approaches}

\section{Specific Formulations}
\subsection{The Continuous-Galerkin PEM}
\subsection{The Weak-Galerkin PEM}
\subsection{The Discontinuous-Galerkin PEM}

\section{Enhancements to Improve Solution Accuracy}
\subsection{Partition-Based Enhancement Functions}
\subsection{Mixed PEM Discretizations}
