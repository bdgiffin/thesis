\chapter{Partitioned Element Methods} \label{ch:pem}
%
This chapter defines a general class of polytopal element formulations referred to as partitioned element methods (PEM). The essential characteristics and mathematical requirements placed upon these methods are formally stated, giving rise to a family of different approaches, for which some formal investigations are conducted. Several specific formulations are summarized in detail, and a number of existing methods are herein classified as particular instances of partitioned element methods.

\section{Overview}

	% goal is to construct shape functions on arbitrary polytopes
	% invoke idea behind FE approximation spaces: use discretization
	% solve local approximation problem to define SFs
A partitioned element method is a finite element-like method which approaches the task of constructing shape functions on an arbitrary polytopal element domain by partitioning the element into simpler sub-domains (quadrature cells), and establishing a local approximation space defined on this partition. Optimal shape functions which minimize a chosen objective functional and which respect a set of imposed constraints are selected from this approximation space.

Partitioned element methods are motivated by the idea that it is generally easier and more efficient to define complicated functions over arbitrary polytopal domains if the functions are defined in a piece-wise polynomial fashion over simpler sub-domains. This is precisely the mentality which likewise motivates the finite element method.

	% want efficient and stable quadrature rules
The PEM is driven by the need for establishing stable and efficient quadrature rules on arbitrary polytopes. Unlike virtual element methods which typically circumvent the use of quadrature altogether, partitioned elements recognize the necessity of domain quadrature rules for evaluating nonlinear residual and stiffness terms. The use of sufficient quadrature also naturally yields a stable integration of the weak form which does not rely upon unphysical stabilization parameters.

	% depart from the idea of shape functions defined pointwise
In contrast with most traditional perspectives which regard the shape functions as being continuously defined on element domains, the PEM exploits the fact that the shape functions and their gradients only need to be evaluated at a discrete number of quadrature points. With this in mind, the PEM constructs an approximation space which is centered around the quadrature cell partition of the element, and defines the shape functions within this discrete space.

	% enforce constraints to attain reproducibility & consistency
The shape functions of each element are subject to the conditions of approximability, compatibility, stability, and quadrature consistency, as detailed in \ref{ch:solid_mechanics}. Together, these conditions impose a number of unique requirements upon the element's partition and the associated cell-based approximation spaces. The resulting shape functions are obtained as the unique solutions to a set of constrained optimization problems, which arise primarily from the conditions of weak compatibility.

%PEM encompasses the special case where each element consists of a single sub-domain. Such approaches will henceforth be referred to as \textit{single-cell} methods, which share certain similarities with the VEM and VETFEM. Single-cell methods are not particularly robust when applied to arbitrary polyhedral meshes (namely those generated by B-rep intersection); they perform poorly if the element's geometry is sufficiently complicated.

In the following sections, an abstract framework for the PEM is established, and a formal statement of several variational PEM problems are detailed.

\section{Shape Functions as Unique Solutions to \\ Boundary Value Problems}

The idea of element shape functions being defined as the solution to a corresponding boundary value problem (in particular, as the solution to Laplace's equation) may be traced back to the work of Gordon and Wixom \cite{Gordon:74}. This concept was later applied to polyhedral finite element methods by Martin et al. in \cite{Martin:08}. Likewise, many virtual element methods suppose that the element shape functions are harmonic, though they are never explicitly constructed or represented as such.

The notion that the shape functions define a set of smooth and continuous interpolants on an element is an appealing concept, though it can frequently lead to numerical complications (nearly-singular solutions), particularly if the elements possess degenerate geometric features. Herein we postulate a number of alternative boundary value problems for constructing shape functions on arbitrary polytopes. These formulations are informed by the conditions of \textit{weak compatibility}, discussed in the following section.

\subsection*{Weak Compatibility}

Consider the necessary and sufficient conditions for compatibility of a deformation gradient field $\mathbf{F}$ stated in equation (\ref{eq:compatibility}), reiterated:
\begin{equation}
	\nabla_X \times \mathbf{F} = \mathbf{0} \quad \forall \mathbf{X} \in \mathcal{B}_0.
\end{equation}
If the deformation gradient is computed from a set of basis functions $\left\{ \varphi_a \right\}_{a=1}^N$ for the displacement field via
\begin{equation}
	\mathbf{F} = \sum_{a=1}^N \mathbf{x}_{a} \otimes \nabla_X \varphi_{a},
\end{equation}
we arrive at a set of equivalent conditions on the basis functions:
\begin{equation}
	\nabla_X \times \nabla_X \varphi = \mathbf{0} \quad \forall \mathbf{X} \in \mathcal{B}_0, \, \varphi \in \left\{ \varphi_a \right\}_{a=1}^N.
	\label{eq:strong_compatibility}
\end{equation}
These equations are trivially satisfied when the basis functions are $C^2 (\mathcal{B}_0)$ continuous, and may therefore be used to construct admissible trial solutions to the strong form statement of equilibrium described in section \ref{sec:strongform}.

In general, the construction of basis functions $\varphi \in C^2 (\mathcal{B}_0)$ is not trivial. Moreover, we seek solutions to the weak form such that $\mathbf{u} \in H^1 (\Omega)$. In what follows, we shall consider the space of functions $\varphi \in C^0 (\mathcal{B}_0)$ satisfying a weakened statement of compatibility:
\begin{equation}
	\int_{\mathcal{B}_0} (\nabla_X \times \nabla_X \varphi) \cdot \mathbf{v} \, dV = 0 \quad \forall \mathbf{v} \in C^\infty_0 (\mathcal{B}_0).
\end{equation}
Through repeated application of integration by parts, the condition that the test functions satisfy compatibility ($\nabla_X \times \nabla_X \mathbf{v} = \mathbf{0} \, \forall \mathbf{v} \in C^\infty_0 (\mathcal{B}_0)$), and the divergence theorem, we determine
% e_{ijk} \varphi_{,ji} v_{k} = (e_{ijk} \varphi_{,j} v_{k})_{,i} + e_{ijk} \varphi_{,j} v_{i,k}
% e_{ijk} \varphi_{,j} v_{i,k} = (e_{ijk} \varphi v_{i,k})_{,j} - e_{ijk} \varphi v_{i,kj} = (e_{ijk} \varphi v_{i,k})_{,j}
% e_{ijk} \varphi_{,ji} v_{k} = (e_{ijk} \varphi_{,j} v_{k})_{,i} + (e_{ijk} \varphi v_{k,j})_{,i} = (e_{ijk} (\varphi v_{k})_{,j})_{,i}
 \begin{equation}
	\int_{\mathcal{B}_0} \nabla_X \cdot (\nabla_X \varphi \times \mathbf{v}) \, dV + \int_{\mathcal{B}_0} \nabla_X \varphi \cdot (\nabla_X \times \mathbf{v}) \, dV = 0,
\end{equation}
\begin{equation}
	\int_{\mathcal{B}_0} \nabla_X \cdot (\nabla_X \times (\varphi \mathbf{v})) \, dV = \int_{\partial \mathcal{B}_0} (\nabla_X \times (\varphi \mathbf{v})) \cdot d\mathbf{A} = 0.
\end{equation}

If we consider a partition $\mathcal{T}^h = \left\{ \Omega_e \right\}_{e=1}^{N_e}$ of $\mathcal{B}_0$ into polytopal elements $\Omega_e \subset \mathcal{B}_0$ whose shape functions are assumed to be piece-wise smooth and continuous on element interiors, then we may impose the equivalent condition:
\begin{equation}
	\sum_{e = 1}^{N_e} \int_{\partial \Omega_e} (\nabla_X \times (\varphi|_{\Omega_e} \mathbf{v})) \cdot d\mathbf{A} = 0 \quad \forall \varphi, \, \mathbf{v} \in C^\infty_0 (\mathcal{B}_0).
\end{equation}
It it evident that if $[\![ \varphi ]\!] = \varphi|_{\Omega_a} - \varphi|_{\Omega_b} = 0 \, \forall \partial \Omega_a \cap \partial \Omega_b \neq \emptyset$, then the above condition is trivially satisfied. Such is the case for $C^0 (\mathcal{B}_0)$ conforming finite element methods.

If $\varphi \not\subset C^0 (\mathcal{B}_0)$, we require
\begin{equation}
	\lim_{h \rightarrow 0} \sum_{e = 1}^{N_e} \int_{\partial \Omega_e} (\nabla_X \times (\varphi|_{\Omega_e} \mathbf{v})) \cdot d\mathbf{A} = 0 \quad \forall \varphi, \, \mathbf{v} \in C^\infty_0 (\mathcal{B}_0),
\end{equation}
which is equivalent to
\begin{equation}
	\lim_{h \rightarrow 0} \sum_{e = 1}^{N_e} \int_{\partial \Omega_e} (\varphi|_{\Omega_e} (\nabla_X \times \mathbf{v}) + \nabla_X \varphi|_{\Omega_e} \times \mathbf{v}) \cdot d\mathbf{A} = 0 \quad \forall \varphi, \, \mathbf{v} \in C^\infty_0 (\mathcal{B}_0).
\end{equation}
Equivalently, we may enforce this condition over each individual face $\Gamma = \Omega_a \cap \Omega_b \neq \emptyset$, such that
\begin{equation}
	\lim_{h \rightarrow 0} \int_{\Gamma} ([\![ \varphi ]\!] (\nabla_X \times \mathbf{v}) + \nabla_X [\![ \varphi ]\!] \times \mathbf{v}) \cdot d\mathbf{A} = 0 \quad \forall \Gamma, \, \mathbf{v} \in C^\infty_0 (\mathcal{B}_0).
\end{equation}
We may expand $[\![ \varphi ]\!] = a_0 + \mathbf{a}_1 \cdot \mathbf{X} + O(\mathbf{X}^2)$ and $\mathbf{v} = \mathbf{b}_0 + \mathbf{B}_1 \mathbf{X} + O(\mathbf{X}^2)$ in terms of low-order polynomials to illustrate
\begin{equation}
	\lim_{h \rightarrow 0} \int_{\Gamma} ((a_0 + \mathbf{a}_1 \cdot \mathbf{X}) \mathbf{B}_1 + \mathbf{a}_1 \times (\mathbf{b}_0 + \mathbf{B}_1 \mathbf{X})) \cdot d\mathbf{A} = 0 \quad \forall \Gamma, \, \mathbf{v} \in C^\infty_0 (\mathcal{B}_0).
\end{equation}

\begin{equation}
	\lim_{h \rightarrow 0} \sum_{e = 1}^{N_e} \int_{\partial \Omega_e} (\varphi|_{\Omega_e} (\nabla_X \times \mathbf{v})) \cdot d\mathbf{A} - \sum_{e = 1}^{N_e} \int_{\Omega_e} \nabla_X \varphi|_{\Omega_e} \cdot (\nabla_X \times \mathbf{v}) \, dV = 0 \quad \forall \varphi, \, \mathbf{v} \in C^\infty_0 (\mathcal{B}_0),
\end{equation}
and because $\nabla_X \times : C^\infty_0 (\mathcal{B}_0) \mapsto C^\infty_0 (\mathcal{B}_0)$, we have
\begin{equation}
	\lim_{h \rightarrow 0} \sum_{e = 1}^{N_e} \int_{\partial \Omega_e} \varphi|_{\Omega_e} \mathbf{v} \cdot d\mathbf{A} - \sum_{e = 1}^{N_e} \int_{\Omega_e} \mathbf{v} \cdot \nabla_X \varphi|_{\Omega_e} \, dV = 0 \quad \forall \varphi, \, \mathbf{v} \in C^\infty_0 (\mathcal{B}_0).
\end{equation}
Recognizing that
\begin{equation}
	\int_{\Omega_e} \mathbf{v} \cdot \nabla_X \varphi|_{\Omega_e} \, dV \leq  || \nabla_X \varphi|_{\Omega_e} ||^2_{\Omega_e},
\end{equation}

If we insist on
\begin{equation}
	\sum_{e = 1}^{N_e} \int_{\partial \Omega_e} (\nabla_X \times (\varphi|_{\Omega_e} \mathbf{v})) \cdot d\mathbf{A} = 0 \quad \forall \varphi, \, \mathbf{v} \in P^1 (\mathcal{B}_0),
\end{equation}
i.e. if we expand $\mathbf{v} = \mathbf{v}_0 + \mathbf{V}_1 \mathbf{x}$ and 
\begin{equation}
	\sum_{e = 1}^{N_e} \int_{\partial \Omega_e} (\nabla_X \times (\varphi|_{\Omega_e} \mathbf{v})) \cdot d\mathbf{A} = 0 \quad \forall \varphi, \, \mathbf{v} \in P^1 (\mathcal{B}_0),
\end{equation}

\section{The Partitioned Element Framework}

\subsection*{The Element Partition}

	%consider an arbitrary element domain
	%discretize domain into conforming sub-domains: a partition
	%limitation: sub-domains must be polytopes
	%shape constraints imposed according to chosen approx. method

\begin{figure} [!ht]
	\centering
	\includegraphics[width = 4.0in]{figures/partition.png}
	\caption{A representative domain $\Omega \subset \mathbb{R}^2$, and it's corresponding partition into verticies, segments, and facets.}
	\label{fig:partitioned_element}
\end{figure}

\begin{itemize}
	\item Provide a figure of an element with partitioned geometry
	\item Define partitioned geometry terms (cells, facets, segments, verticies, etc.), perhaps even give a table of definitions, all referring back to the main figure(s)
	\item 
\end{itemize}

\subsection{Partition-Based Approximation Spaces}

		%define SFs as FE-like piece-wise polynomials on the partition
		%try to minimize number of basis functions for efficiency
		\subsubsection*{FE basis}
			%unknowns stored at verticies only (efficient)
			%easy generalization to high order
			%limited to canonical shapes (tris/tets, quads/hexas)
			%quadrature rules much simpler (may use high order)
			%equivalent to Joe Bishop's approach (w/ Laplace SFs)
			%less sensitive to conditioning problems
		\subsubsection*{WG basis}
			%unknowns stored on cells and edges
			%generalization to high order more expensive
			%Rashid & Sadri 2012 approach (low order)
			%may suffer from conditioning issues at high order
		\subsubsection*{DG basis}
			%unknowns stored on cells
			%easy generalization to high order
			%New/current approach
			%may suffer from conditioning issues at high order
		\subsubsection*{VE basis}
			%unknowns stored at verticies only (efficient)
			%high order generalization effected via serendipity SFs
			%generalizes to arbitrary polytopes
			%New/speculated approach
			%less sensitive to conditioning problems

Traditional approximation methods typically consider the independent specification of two principal transformations: an interpolation scheme, which is necessary to represent field variables according to known point-values; and a quadrature rule, for the purposes of integrating such fields over the domain on which they are defined.

In mathematical terms, an interpolant $\varphi$ is a linear operator which maps vectors $\mathbf{u} \in \mathbb{R}^k$ containing point-wise data regarding a field into scalar functions $f \in V^k (\Omega)$, i.e.
\begin{equation}
  \varphi \colon \mathbb{R}^k \mapsto V^k (\Omega)
\end{equation}
where $\mathbb{R}^k$ is a $k$-dimensional real space, and $V^k (\Omega)$ is a $k$-dimensional function space defined on $\Omega$.

By constrast, a quadrature rule $\Sigma$ is a linear operator which maps scalar functions $f \in V^k (\Omega)$ to vectors $\mathbf{q} \in \mathbb{R}^p$ identifying point-wise samples of $f$, i.e.
\begin{equation}
  \Sigma \colon V^k (\Omega) \mapsto \mathbb{R}^p
\end{equation}

The composition $\Sigma \circ \varphi$ of an interpolant $\varphi$ with a quadrature rule $\Sigma$ yields a linear operator
\begin{equation}
  \Sigma \circ \varphi \colon \mathbb{R}^k \mapsto \mathbb{R}^p
\end{equation}

\subsection{Partition-Based Quadrature Rules}
		%quadrature rules are defined/linked to the chosen partition
		%1-to-1 relation between partitioned cells and quadrature points
		%use Lassere integration to get weights of polytopal domains
		%employ low-order quadrature rules for the sake of efficiency
		%FE/simplicial discretization allows high-order composite rules
		
	\subsubsection*{Composite Midpoint Quadrature}
	
	
		
	\subsubsection*{Selective Modal Quadrature}
	
	Consider all functions $f \in L^2 (\Omega)$ represented over an arbitrary polytopal element domain $\Omega \subset \mathbb{R}^d$. Standard quadrature rules approximate the integral of $f$ over $\Omega$ as
	\begin{equation}
		\int_\Omega f \, dv \approx \sum_{q=1}^{N_{qp}} w_q f(\mathbf{x}_q).
	\end{equation}
	Consider an $L^2 (\Omega)$ polynomial projection operator $\Pi : L^2 \mapsto P^k$ which may be used to decompose $f = f_p + f_n$ into polynomial and non-polynomial parts:
	\begin{equation}
		f_p = \Pi f, \quad f_n = f - \Pi f = \pi f,
	\end{equation}
	where $\pi : L^2 \mapsto L^2 \backslash P^k$. Consequently, we observe that $\Pi f$ is $L^2$ orthogonal to any $\pi g$ for all $g \in L^2$, to the extent that
	\begin{equation}
		\int_{\Omega} (\Pi f) (g - \Pi g) \, dv = \langle \Pi f, g - \Pi g \rangle_{\Omega} = 0 \quad \forall f, \, g \in L^2.
	\end{equation}
	
	We propose a quadrature rule of the form:
	\begin{equation}
		\int_\Omega f \, dv \approx \int_\Omega f_p \, dv + \sum_{q=1}^{N_{qp}} w_q f_n(\mathbf{x}_q),
	\end{equation}
	where it is supposed that $\int_\Omega f_p \, dv$ may be computed exactly using the methodology proposed by Lasserre in \cite{Chin:15}. Further, if we wish to integrate the product $f g$ where $f, \, g \in H^1 (\Omega)$, we may write
	\begin{equation}
		\int_\Omega f g \, dv = \langle f, g \rangle_{\Omega} = \langle \Pi f + \pi f, \Pi g + \pi  g \rangle_{\Omega},
	\end{equation}
	which, by the linearity of the $L_2$ inner product, and by the orthogonality of $\Pi f$ and $\pi g$ (and of $\pi f$ and $\Pi g$), yields
	\begin{equation}
		\int_\Omega f g \, dv = \langle \Pi f, \Pi g \rangle_{\Omega} + \langle \pi f, \pi g \rangle_{\Omega},
	\end{equation}
	and thus
	\begin{equation}
		\int_\Omega f g \, dv \approx \int_\Omega f_p g_p \, dv + \sum_{q=1}^{N_{qp}} w_q f_n(\mathbf{x}_q) g_n(\mathbf{x}_q).
	\end{equation}
	This is effectively equivalent to integrating the product of all low-order polynomials exactly, while integrating the product of all non-polynomial ``remainders'' only approximately, using a quadrature rule.
	
	If we are only given point evaluations of a function $f$ at $\left\{ \mathbf{x}_q \right\}_{q=1}^{N_{qp}}$, then we must construct a low-order polynomial projection operator by considering the least-squares problem:
	\begin{equation}
		\min_{f_p \in P^k (\Omega)} \frac{1}{2} || f_p - f ||^2_\Omega,
	\end{equation}
	where $|| f ||_\Omega = \sqrt{\langle f, f \rangle_\Omega}$ is deliberately approximated using the element's quadrature rule:
	\begin{equation}
		\langle f, f \rangle_\Omega \approx \sum_{q=1}^{N_{qp}} w_q \left[ f(\mathbf{x}_q) \right]^2.
		\label{eq:L2projection}
	\end{equation}
	For a given polynomial basis $\left\{ z_a \right\}_{a=1}^{K}$ which spans $P^k (\Omega)$, we may write
	\begin{equation}
		f_p (\mathbf{x}) = \sum_{a=1}^{K} z_a (\mathbf{x}) c_a = \mathbf{z}^T (\mathbf{x}) \, \mathbf{c},
	\end{equation}
	and the solution to (\ref{eq:L2projection}) satisfies
	\begin{equation}
		\sum_{a=1}^{K} \sum_{q=1}^{N_{qp}} w_q z_b (\mathbf{x}_q) z_a (\mathbf{x}_q) c_a =
		\sum_{q=1}^{N_{qp}} w_q z_b (\mathbf{x}_q) f(\mathbf{x}_q) \quad \forall b = 1, \, \ldots, \, K,
	\end{equation}
	which may be written in matrix form as $\mathbf{Z}^T \mathbf{W} \mathbf{Z} \mathbf{c} = \mathbf{Z}^T \mathbf{W} \mathbf{f}$, where we denote $f_i = f(\mathbf{x}_i)$, $W_{ii} = w_i, \, W_{ij} = 0 \, \forall i \neq j$, and $Z_{ij} = z_j (\mathbf{x}_i)$. The discrete polynomial projection operator $\mathbf{\Pi} : \mathbb{R}^{N_{qp}} \mapsto \mathbb{R}^K$ is computed as $\boldsymbol{\Pi} = (\mathbf{Z}^T \mathbf{W} \mathbf{Z})^{-1} \mathbf{Z}^T \mathbf{W}$, and the complement operator $\boldsymbol{\pi} : \mathbb{R}^{N_{qp}} \mapsto \mathbb{R}^{N_{qp}}$ is $\boldsymbol{\pi} = \mathbf{1}_{N_{qp}} - \boldsymbol{\Pi}^\dagger \boldsymbol{\Pi}$. Consequently,
	\begin{equation}
		\langle \Pi f, \Pi g \rangle_{\Omega} = \langle \mathbf{z}^T \boldsymbol{\Pi} \mathbf{f}, \mathbf{z}^T \boldsymbol{\Pi} \mathbf{g} \rangle_{\Omega} = \mathbf{f}^T \left[ \boldsymbol{\Pi}^T \left( \int_{\Omega} \mathbf{z} \otimes \mathbf{z} \, dv \right) \boldsymbol{\Pi} \right] \mathbf{g} = \mathbf{f}^T \mathbf{W}_p \mathbf{g},
	\end{equation}
	\begin{equation}
		\langle \pi f, \pi g \rangle_{\Omega} \approx \sum_{q=1}^{N_{qp}} w_q f_n(\mathbf{x}_q) g_n(\mathbf{x}_q) = \mathbf{f}^T \left[ \boldsymbol{\pi}^T \mathbf{W} \boldsymbol{\pi} \right] \mathbf{g} = \mathbf{f}^T \mathbf{W}_n \mathbf{g},
	\end{equation}
	\begin{equation}
		\int_\Omega f g \, dv \approx \mathbf{f}^T (\mathbf{W}_p + \mathbf{W}_n) \mathbf{g} = \mathbf{f}^T \mathbf{M} \mathbf{g} = \sum_{q = 1}^{N_{qp}} \sum_{p = 1}^{N_{qp}} M (\mathbf{x}_q,\mathbf{x}_p) \, f (\mathbf{x}_q) \, g (\mathbf{x}_p).
	\end{equation}
	
	We shall refer to this form of integration as \textit{selective modal quadrature}, in that particular low-order polynomial modes are integrated exactly, and any higher modes are approximated using the element's quadrature rules. The advantage of modal quadrature is that we may exactly integrate any terms which directly impact quadrature consistency, and hence, virtually any stable quadrature may be used to integrate the higher-order part.
	
	Rather than storing the independent quadrature weights $w_q = w (\mathbf{x}_q)$, selective modal quadrature requires the storage of a generalized quadrature weighting matrix $M (\mathbf{x}_q,\mathbf{x}_p)$. Alternatively, for the sake of efficiency, if the function $g$ is known a priori (e.g. if $g$ is a test function for a weighted residual method), then we need only store the ``augmented'' test function values:
	\begin{equation}
		\tilde{g} (\mathbf{x}_q) = \sum_{p = 1}^{N_{qp}} \frac{M(\mathbf{x}_q,\mathbf{x}_p)}{w_q} \, g (\mathbf{x}_p),
	\end{equation}		
	and all integrals involving $f$ and $g$ may be carried out via
	\begin{equation}
		\int_\Omega f g \, dv \approx \sum_{q = 1}^{N_{qp}} \, w_q \, \tilde{g} (\mathbf{x}_q) \, f (\mathbf{x}_q),
	\end{equation}
	which is effectively equivalent to a gradient correction scheme, similar to the method proposed in \cite{Talischi:15}.

\subsection{Selection of an Appropriate Objective Functional}

	%PEM BOUNDARY VALUE PROBLEMS
		%setup: define a space of approximating functions over the element (basis)
		%goal:  select shape functions from this basis that are "optimal"
			%trial functions should reproduce polynomials as best they can
			%test functions should be close to trial functions (for stability)
			%test functions should satisfy compatibility requirements weakly
			%test functions should satisfy quad. consistency requirements against polys.
		%must propose a positive-definite functional to minimize on E
		%must enforce constraints on minimization
			%consistency enforced via Lagrange multipliers
		%develop a resulting system of equations
		%condense out all internal dofs in terms of element nodal values
		%allows for the inclusion of optional enhanced dofs, if needed
		%want SFs to be free of pathologies
		%Laplace shape functions
			%Joe Bishop's approach
			%Rashid & Sadri 2012 approach
		%Least-squares penalty method
			%necessary to stabilize WG/DG approaches
		%Other methods
			%many things to explore in this regard...
			%biharmonic? (for higher-order completeness? C^1 hard)

Once we have specified a given approximation space of functions, our goal is then to select the function from this space which best represents the nodal data that we are attempting to interpolate over the element. The essential question is this: by what metric should we objectively assess the appropriateness of a given approximating function?

\subsection*{Construction of Element Appoximants}

\section{Essential Requirements of the PEM}

\subsection*{Reproducibility}

Fundamentally, the approximation power of the PEM depends directly upon the degree of completeness of the underlying approximation space. In particular, the finite-dimensional trial solution space $\mathcal{S}^h$ should contain as a subspace $\mathbb{P}_k \subset \mathcal{S}^h$ for some $k \geq 0$, where $\mathbb{P}_k$ denotes the subspace of polynomial functions with maximal degree $k$. This guarantees that the PEM approximation space will be capable of exactly reproducing any polynomial function up to some specified order.

	%driven by completeness requirements
	%handled by BVP/functional specification
	%alternatively enforced via constraints

\subsubsection*{Requirements of the Approximation Space}
\subsection*{Compatibility}

	%driven by the generalized patch test
	%handled by constraints on approximation spaces/BVP



\subsubsection*{Weak Compatibility}
\subsection*{Stability}

		%driven by the inf-sup conditions
		%handled via sufficient quadrature arrangement

\subsubsection*{Restrictions on the Element Partition}
\subsection*{Variational Consistency}
\subsubsection*{Weak Enforcement of Consistency}

		%driven by Galerkin exactness
		%enforced by quadrature accuracy
		%alternatively via lagrange multiplier constraint

Recall the $\alpha$-th order variational consistency requirements:
\begin{equation}
  \int_\Omega \mathbf{x}^{\alpha} \varphi_{a,i} \, dv + \int_\Omega \mathbf{x}^{\alpha}_{,i} \varphi_a \, dv = \int_{\partial \Omega} \mathbf{x}^{\alpha} n_i \varphi_a \, da \quad \forall i, a.
\end{equation}
This degree of consistency is required for all $| \alpha | \leq k-1$ if the bi-linear form $a : \mathcal{S} \times \mathcal{V} \mapsto \mathbb{R}$ is to determine uniqueness of any exact polynomial solutions up to order $k$ contained within $\mathcal{S}^h \supset P_k$. Moreover, the above expression implies that any solution may be reasonably well approximated by low-order polynomials in the locality of the selected domain $\Omega$.

Consequently, we may consider an application of the above consistency requirements to the entire problem domain $\Omega$, or -- more practically -- we may enforce consistency up to the desired order on each element domain $\Omega_e$. The element-local $\alpha$-th order consistency equations follow:
\begin{equation}
  \int_{\Omega_e} \mathbf{x}^{\alpha} \varphi_{a,i} \, dv + \int_{\Omega_e} \mathbf{x}^{\alpha}_{,i} \varphi_a \, dv = \int_{\partial \Omega_e} \mathbf{x}^{\alpha} n_i \varphi_a \, da \quad \forall i, a.
  \label{eq:consistency3}
\end{equation}

In most practical situations, it may not be possible to evaluate the integral expressions exactly. Instead, numerical quadrature rules must be defined, both on the element's interior, and on its boundary, such that
\begin{equation}
  \int_{\Omega_e} f(\mathbf{x}) dv \approx \sum_{q=1}^{N_q} w_q f(\mathbf{x}_q), \quad \int_{\partial \Omega_e} f(\mathbf{x}) da \approx \sum_{b=1}^{N_b} w_b f(\mathbf{x}_b).
\end{equation}
This yields yet another form of consistency, henceforth referred to as ``quadrature consistency,'' expressed as:
\begin{equation}
  \sum_{q=1}^{N_q} w_q \left[ \mathbf{x}_q^{\alpha} \varphi^{(q)}_{a,i} + \mathbf{x}^{\alpha}_{q,i} \varphi^{(q)}_a \right] = \sum_{b=1}^{N_b} w_b \mathbf{x}_b^{\alpha} n^{(b)}_i \varphi^{(b)}_a \quad \forall i, a.
  \label{eq:quadrature_consistency3}
\end{equation}
Moving forward in our developments, quadrature consistency will be the only form of consistency of practical interest, as it appropriately reflects the discrete data quantities at our disposal.

A brief remark should be made regarding the lack of equivalence between the exact evaluation of consistency in equation (\ref{eq:consistency3}) and its numerical approximation in equation (\ref{eq:quadrature_consistency3}). In general, we cannot claim that satisfaction of one guarantees the other. There is only one occasion when we may say that (\ref{eq:quadrature_consistency3}) holds if and only if (\ref{eq:consistency3}) is satisfied, which occurs only when sufficiently accurate quadrature rules are used on the element and its boundary. In the vast majority of situations, however, this is an impractical constraint, as the construction of high-order quadrature rules on arbitrary polyhedra (though possible) proves to be somewhat computationally intensive (see the literature by Sukumar, etc.). Moreover, without a precise representation of the test functions over the element domain, it becomes infeasible to consider an exact integration of such functions.

\subsection*{Numerical Quadrature Consistency}

Beyond consistency with the weak form for a specified subspace of polynomial functions used to represent the solution, we also require that the numerical quadrature scheme utilized be sufficiently accurate to the extent that: Galerkin exactness is satisfied for certain classes of solutions, and that the quadrature error has a bound which is of a lower order than that of the approximation error.

\subsubsection*{Bubnov-Galerkin and Petrov-Galerkin Approaches}

\section{Specific Formulations}

\subsection*{The Continuous-Galerkin PEM}

	% Harmonic SFs with an FE basis (Joe Bishop's approach)

\subsection*{The Weak-Galerkin PEM}

	% Stabilized harmonic SFs with WG basis (Rashid & Sadri approach)

\subsection*{The Discontinuous-Galerkin PEM}

	% C^k penalty approximating SFs with DG basis (new method)
	
\subsection*{PEM Based on Composite Virtual Elements}

	% Harmonic SFs with VE basis + stabilization (new method)

\section{Enhancements to Improve Solution Accuracy}

		%GENERAL RULE: enhancements should seek to increase completeness
			%witnessed in most incompatible modes formulations
			%only way to improve accuracy: increase order, locally
		%balance stability, consistency, compatibility, reproducibility
			%bubble function construction closely tied to other SFs
			%enhancements must be orthogonal to low-order terms
			%must satisfy compatibility weakly over whole boundary
		%element-wide high order enhancement functions
			%inexpensive way to improve polynomial completeness
			%may apply to boundary dofs, or internal dofs

\subsection*{Partition-Based Enhancement Functions}
\subsection*{Mixed PEM Discretizations}
