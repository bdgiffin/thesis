\chapter{An Overview of Computational Solid Mechanics}
%
Justify a methodology that suitably addresses all of the issues discussed herein. A method which could solve all of these problems effectively likely does not exist, but may be ideal for a particular application.

Stipulate the essential requirements for that an approximation method must satisfy, setting up the subsequent discussion

\section{The Lagrangian Description of Motion}

Consider a body $\mathcal{B}$ consisting of a set of material points whose positions in some reference configuration at time $t = 0$ are denoted by $\mathbf{X} \in \mathbb{R}^d$. The motion of individual material points in the body will yield new spatial positions $\mathbf{x}$ at time $t > 0$ defined by the bijection $\mathcal{X} \colon \mathbf{X} \mapsto \mathbf{x}$, i.e.
\begin{equation}
  \mathbf{x} = \mathcal{X} (\mathbf{X}, t).
\end{equation}
The displacement $\mathbf{u}$ of a given material point may be expressed as $\mathbf{u} = \mathbf{x} - \mathbf{X}$, and it's corresponding velocity may be written $\mathbf{v} = \dot{\mathbf{x}}$.

At a given time $t$, the Jacobian of the deformation mapping $\mathcal{X}$ yields the deformation gradient $\mathbf{F}$ (a rank-2 tensor), defined as
\begin{equation}
  \mathbf{F} = \frac{\partial \mathbf{x}}{\partial \mathbf{X}} = \mathbf{1} + \nabla_X \mathbf{u},
\end{equation}
where $\nabla_X$ denotes the gradient with respect to $\mathbf{X}$. The deformation gradient may be used to map differential line segments $d \mathbf{X}$, surface areas $d \mathbf{A}$, and volumes $d V$ defined in the reference configuration into their corresponding transformed quantities ($d \mathbf{x}$, $d \mathbf{a}$, $dv$) defined in the current configuration at time $t$:
\begin{equation}
  d \mathbf{x} = \mathbf{F} \cdot d \mathbf{X}
\end{equation}
\begin{equation}
  d \mathbf{a} = J \mathbf{F}^{-T} \cdot d \mathbf{A}
\end{equation}
\begin{equation}
  d v = J d V
\end{equation}
where $J \equiv \det{\mathbf{F}}$.

In a similar fashion, the spatial velocity gradient $\mathbf{L} = \nabla_x \mathbf{v}$ (where $\nabla_x$ denotes the gradient with respect to $\mathbf{x}$) may be expressed as
\begin{equation}
  \mathbf{L} = \frac{\partial \mathbf{v}}{\partial \mathbf{x}} = \frac{\partial \dot{\mathbf{x}}}{\partial \mathbf{X}} \frac{\partial \dot{\mathbf{X}}}{\partial \mathbf{x}} = \dot{\mathbf{F}} \mathbf{F}^{-1},
\end{equation}
which may be further decomposed into a symmetric part $\mathbf{D}$ (the rate of deformation tensor),
\begin{equation}
  \mathbf{D} = \frac{1}{2} (\mathbf{L} + \mathbf{L}^T),
\end{equation}
and an anti-symmetric part $\mathbf{W}$ (the spin tensor),
\begin{equation}
  \mathbf{W} = \frac{1}{2} (\mathbf{L} - \mathbf{L}^T).
\end{equation}


\section{Strong Formulation of the Model Solid Mechanics Boundary Value Problem}

The study of solid mechanics principally deals with the conservation of linear and angular momentum, such that a given body with appropriately defined boundary conditions is rendered to be in force equilibrium. A fundamental assertion to that end is that for a given body $\mathcal{B}$ in equilibrium, any open subset $\Omega \subset \mathcal{B}$ is likewise in equilibrium, such that the net force which acts upon $\Omega$ is equal to zero (according to Newton's first law), i.e.
\begin{equation}
  \int_{\Omega} \rho (\mathbf{b} - \dot{\mathbf{v}}) \, dv + \int_{\partial \Omega} \mathbf{t} \, da = \mathbf{0} \quad \forall \Omega \subset \mathcal{B},
\end{equation}
where $\rho$ is the specific mass density of the matertial, $\mathbf{b}$ is an applied body force per unit of mass, $\dot{\mathbf{v}}$ is the linear acceleration field (a function of position, $\mathbf{x}$), and $\mathbf{t}$ is the traction vector (force per unit of area) which acts on $\partial \Omega$. Via the Cauchy tetrahedron argument, it is possible to express the traction vector $\mathbf{t} (\mathbf{n})$ as a linear function of the surface normal $\mathbf{n}$ upon which the traction acts:
\begin{equation}
  \mathbf{t} = \mathbf{n} \cdot \mathbf{T},
\end{equation}
where $\mathbf{T}$ is referred to as the Cauchy stress tensor. Invoking the divergence theorem, we may utilize the above relation to convert the traction boundary integral into a volume integral over $\Omega$:
\begin{equation}
  \int_{\partial \Omega} \mathbf{t} \, da = \int_{\partial \Omega} \mathbf{n} \cdot \mathbf{T} \, da = \int_{\Omega} \nabla \cdot \mathbf{T} \, dv,
\end{equation}
which yields
\begin{equation}
  \int_{\Omega} \left[ \rho (\mathbf{b} - \dot{\mathbf{v}}) + \nabla \cdot \mathbf{T} \right] \, dv = \mathbf{0} \quad \forall \Omega \subset \mathcal{B}.
\end{equation}
Since we have imposed no limitations on the choice of subset $\Omega$, we may invoke the localization theorem to determine a point-wise statement of equilibrium in the body $\mathcal{B}$:
\begin{equation}
  \rho (\mathbf{b} - \dot{\mathbf{v}}) + \nabla \cdot \mathbf{T} = \mathbf{0} \quad \forall \mathbf{x} \in \mathcal{B}.
\end{equation}
Formulating an expression for the conservation of angular momentum and following a similar procedure, we obtain an additional point-wise requirement:
\begin{equation}
  \mathbf{T} = \mathbf{T}^T \quad \forall \mathbf{x} \in \mathcal{B}.
\end{equation}

For the model problem in question, the body itself is assumed to satisfy compatibility, such that the displacement field $\mathbf{u}$ is a $C^0$ continuous field (i.e. the topology of the body is fixed). The evolution of the stress field shall be related to the motion of the body, such that:
\begin{equation}
  \accentset{\circ}{\mathbf{T}} = f (\mathbf{T}, \mathbf{D}, q_*),
\end{equation}
where $\mathbf{D}$ specifies the rate of deformation of a given material point, $q_*$ comprise a set of material state variables which fully define the current state of the material, and $\accentset{\circ}{\mathbf{T}}$ denotes an appropriately defined corotational rate of the Cauchy stress (e.g. the Jaumann rate of stress).

\section{The Equivalent Weak (Variational) Formulation of the Model BVP}

\section{Galerkin Approximations to the Weak Form}
\subsection{Symmetric (Bubnov-) Galerkin Methods}
\subsection{Petrov-Galerkin Methods}

\section{Lagrangian Discretizations of the Problem Domain}
\subsection{Conforming Discretizations}
\subsubsection{Finite Element Methods}
\subsubsection{Mesh Free Methods}
\subsection{Non-Conforming Discretizations}

\section{Requirements for Convergence of an Approximation Method}
\subsection{Satisfaction of Essential Boundary Conditions}
\subsection{Completeness/Approximability}
\subsection{Variational Consistency}
\subsubsection{Constraints on Numerical Quadrature}
\subsection{Numerical Stability}
\subsubsection{The Inf-Sup Condition}
\subsection{Tests for Convergence}
\subsubsection{The Irons Patch Test}
\subsubsection{The Generalized Patch Test}
\subsubsection{The F-E-M Test}

\section{Sources of Solution Nonlinearity in Solid Mechanics}
\subsection{Finite Deformations and Nonlinear Kinematics}
\subsection{Nonlinear Material Behavior}
\subsection{Requirements of the Approximation Method}

\section{Additional Considerations}

\subsection{Enforcement of Constraints}
\subsubsection{Contact}
\subsubsection{Mesh Tying and Multi-Point Constraints}

\subsection{Fracture and Topology Change}

\subsection{Solution Remapping}
\subsubsection{Severe Plastic Deformations}
\subsubsection{Mesh Tangling and Inversion}

\subsection{Adaptive Refinement}
\subsubsection{$h$-, $p$-, and $hp$-Adaptivity}
