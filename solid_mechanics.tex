\chapter{An Overview of Computational Solid Mechanics}
%
Justify a methodology that suitably addresses all of the issues discussed herein. A method which could solve all of these problems effectively likely does not exist, but may be ideal for a particular application.

Stipulate the essential requirements for that an approximation method must satisfy, setting up the subsequent discussion 

\section{Strong Formulation of the Solid Mechanics Model Boundary Value Problem}

\section{The Equivalent Weak (Variational) Formulation of the BVP}

\section{Galerkin Approximations to the Weak Form}
\subsection{Symmetric (Bubnov-) Galerkin Methods}
\subsection{Petrov-Galerkin Methods}

\section{Lagrangian Discretizations of the Problem Domain}
\subsection{Conforming Discretizations}
\subsubsection{Finite Element Methods}
\subsubsection{Mesh Free Methods}
\subsection{Non-Conforming Discretizations}

\section{Requirements for Convergence of an Approximation Method}
\subsection{Satisfaction of Essential Boundary Conditions}
\subsection{Completeness/Approximability}
\subsection{Variational Consistency}
\subsubsection{Constraints on Numerical Quadrature}
\subsection{Numerical Stability}
\subsubsection{The Inf-Sup Condition}
\subsection{Tests for Convergence}
\subsubsection{The Irons Patch Test}
\subsubsection{The Generalized Patch Test}
\subsubsection{The F-E-M Test}

\section{Sources of Solution Nonlinearity in Solid Mechanics}
\subsection{Finite Deformations and Nonlinear Kinematics}
\subsection{Nonlinear Material Behavior}
\subsection{Requirements of the Approximation Method}

\section{Additional Considerations}

\subsection{Enforcement of Constraints}
\subsubsection{Contact}
\subsubsection{Mesh Tying and Multi-Point Constraints}

\subsection{Fracture and Topology Change}

\subsection{Solution Remapping}
\subsubsection{Severe Plastic Deformations}
\subsubsection{Mesh Tangling and Inversion}

\subsection{Adaptive Refinement}
\subsubsection{$h$-, $p$-, and $hp$-Adaptivity}
